%% ----------------------------------------------------------------
%% Thesis.tex -- MAIN FILE (the one that you compile with LaTeX)
%% ---------------------------------------------------------------- 

% Set up the document
\documentclass[a4paper, 11pt, oneside]{Tesi}  % Use the "Thesis" style, based on the ECS Thesis style by Steve Gunn
\graphicspath{{Figure/}}  % Location of the graphics files (set up for graphics to be in PDF format)

% Include any extra LaTeX packages required
\usepackage[utf8]{inputenc}
\usepackage[italian]{babel}
\usepackage[T1]{fontenc}
\usepackage{ae,aecompl}
\usepackage[square, numbers, comma, sort&compress]{natbib}  % Use the "Natbib" style for the references in the Bibliography
\usepackage{verbatim}  % Needed for the "comment" environment to make LaTeX comments
\usepackage{vector}  % Allows "\bvec{}" and "\buvec{}" for "blackboard" style bold vectors in maths
\hypersetup{urlcolor=blue, colorlinks=true}  % Colours hyperlinks in blue, but this can be distracting if there are many links.

%% To show a border around all figures
%\usepackage{float}
%\floatstyle{boxed} 
%\restylefloat{figure}

%% ----------------------------------------------------------------
\begin{document}
\frontmatter	  % Begin Roman style (i, ii, iii, iv...) page numbering

% Set up the Title Page
\title  {Reti federate eventualmente connesse}

\authors  {\texorpdfstring
            {\href{mailto:vince@mocambos.net}{Vincenzo Tozzi}}
            {Vincenzo Tozzi}
            }
\addresses  {\groupname\\\deptname\\\univname}  % Do not change this here, instead these must be set in the "Thesis.cls" file, please look through it instead
\date       {\today}
\subject    {Tesi di Laurea in Informatica}
\keywords   {Federated Network, Tecnological Autonomy, Free Software, Rede Mocambos, Casa de Cultura Taina, Quilombo}

\maketitle
%% ----------------------------------------------------------------

\setstretch{1.3}  % It is better to have smaller font and larger line spacing than the other way round

% Define the page headers using the FancyHdr package and set up for one-sided printing
\fancyhead{}  % Clears all page headers and footers
\rhead{\thepage}  % Sets the right side header to show the page number
\lhead{}  % Clears the left side page header

\pagestyle{fancy}  % Finally, use the "fancy" page style to implement the FancyHdr headers

% %% ----------------------------------------------------------------
% % Declaration Page required for the Thesis, your institution may give you a different text to place here
% \Declaration{

% \addtocontents{toc}{\vspace{1em}}  % Add a gap in the Contents, for aesthetics

% Io, Vincenzo Tozzi, dichiaro che la presente tesi, "Reti federate eventualmente connesse" e il lavoro presentato in essa sono di mia paternità. Dichiaro che:

% \begin{itemize} 
% \item[\tiny{$\blacksquare$}] Questo lavoro è stato totalmente o per la maggior parte svolto come laureando per la Laurea in Informatica di questa Università.
 
% \item[\tiny{$\blacksquare$}] Ho esplicitamente dichiarato nel testo se qualche parte di questa tesi è stata precedentemente pubblicata in altri lavori da questa o altre Università o istituzioni.
 
% \item[\tiny{$\blacksquare$}] Ho attribuito la paternità ai lavori pubblicati da altri e da me consultati.
 
% \item[\tiny{$\blacksquare$}] Ho sempre citato la fonte di opere altrui. Ad eccezione di tali citazioni, questa tesi è di mia paternità.
 
% \item[\tiny{$\blacksquare$}] Ho ringranziato tutte le principali fonti di supporto.
 
% \item[\tiny{$\blacksquare$}] Ho esplicitamente dichiarato del testo, in parti sviluppate assieme ad altri, qual'è il loro e il mio contributo.
% \\
% \end{itemize}
 
 
% Firmato:\\
% \rule[1em]{25em}{0.5pt}  % This prints a line for the signature
 
% Data:\\
% \rule[1em]{25em}{0.5pt}  % This prints a line to write the date
% }
% \clearpage  % Declaration ended, now start a new page

%% ----------------------------------------------------------------
% The "Funny Quote Page"
\pagestyle{empty}  % No headers or footers for the following pages

\null\vfill
% Now comes the "Funny Quote", written in italics
\textit{``Facciamo un mondo più a modo nostro''}

\begin{flushright}
Zumbi dos Palmares
\end{flushright}

\vfill\vfill\vfill\vfill\vfill\vfill\null
\clearpage  % Funny Quote page ended, start a new page
%% ----------------------------------------------------------------

% The Abstract Page
\addtotoc{Abstract}  % Add the "Abstract" page entry to the Contents
\abstract{
\addtocontents{toc}{\vspace{1em}}  % Add a gap in the Contents, for aesthetics

This work researches on technologies for a federated network of
afrodescendants communities looking at autonomy and freedom. The work
propose an architecture and prototype for an eventually connected
federated network where the main limit and technical bond is the
connection availability and bandwidth. The prototype is a media
sharing system for local web portals, based on Django and Git-Annex,
authenticated on LDAP servers.

}

\clearpage  % Abstract ended, start a new page
%% ----------------------------------------------------------------

\setstretch{1.3}  % Reset the line-spacing to 1.3 for body text (if it has changed)

% The Acknowledgements page, for thanking everyone
\acknowledgements{
\addtocontents{toc}{\vspace{1em}}  % Add a gap in the Contents, for aesthetics

Un ringraziamento speciale alle mie famiglie\ldots

}
\clearpage  % End of the Acknowledgements
%% ----------------------------------------------------------------

\pagestyle{fancy}  %The page style headers have been "empty" all this time, now use the "fancy" headers as defined before to bring them back


%% ----------------------------------------------------------------
\lhead{\emph{Contents}}  % Set the left side page header to "Contents"
\tableofcontents  % Write out the Table of Contents

%% ----------------------------------------------------------------
\lhead{\emph{List of Figures}}  % Set the left side page header to "List if Figures"
\listoffigures  % Write out the List of Figures

%% ----------------------------------------------------------------
%\lhead{\emph{List of Tables}}  % Set the left side page header to "List of Tables"
%\listoftables  % Write out the List of Tables

%% ----------------------------------------------------------------
\setstretch{1.5}  % Set the line spacing to 1.5, this makes the following tables easier to read
\clearpage  % Start a new page
\lhead{\emph{Abbreviazioni}}  % Set the left side page header to "Abbreviations"
\listofsymbols{ll}  % Include a list of Abbreviations (a table of two columns)
{
% \textbf{Acronym} & \textbf{W}hat (it) \textbf{S}tands \textbf{F}or \\
\textbf{GESAC} & \textbf{G}overno \textbf{E}letrônico \textbf{S}erviço
de \textbf{A}tendimento ao \textbf{C}idadão\\
\textbf{RM} & \textbf{R}ede \textbf{M}ocambos\\
\textbf{SP} & \textbf{S}ervice \textbf{P}rovider\\
\textbf{IdP} & \textbf{Id}entiy \textbf{P}rovider\\

}

%% ----------------------------------------------------------------
%\clearpage  % Start a new page
%\lhead{\emph{Physical Constants}}  % Set the left side page header to "Physical Constants"
%\listofconstants{lrcl}  % Include a list of Physical Constants (a four column table)
%{
%% Constant Name & Symbol & = & Constant Value (with units) \\
%Speed of Light & $c$ & $=$ & $2.997\ 924\ 58\times10^{8}\ \mbox{ms}^{-\mbox{s}}$ (exact)\\
%
%}

%% ----------------------------------------------------------------
%\clearpage  %Start a new page
%\lhead{\emph{Simboli}}  % Set the left side page header to "Symbols"
%\listofnomenclature{lll}  % Include a list of Symbols (a three column table)
%{
%% symbol & name & unit \\
%$a$ & distance & m \\
%$P$ & power & W (Js$^{-1}$) \\
%& & \\ % Gap to separate the Roman symbols from the Greek
%$\omega$ & angular frequency & rads$^{-1}$ \\
%}
%% ----------------------------------------------------------------
% End of the pre-able, contents and lists of things
% Begin the Dedication page

\setstretch{1.3}  % Return the line spacing back to 1.3

\pagestyle{empty}  % Page style needs to be empty for this page
\dedicatory{A Giusy, mia madre\ldots}

\addtocontents{toc}{\vspace{2em}}  % Add a gap in the Contents, for aesthetics


%% ----------------------------------------------------------------
\mainmatter	  % Begin normal, numeric (1,2,3...) page numbering
\pagestyle{fancy}  % Return the page headers back to the "fancy" style

% Include the chapters of the thesis, as separate files
% Just uncomment the lines as you write the chapters

% Capitolo 1

\chapter{Introduzione}
\label{Capitolo1}
\lhead{Capitolo 1. \emph{Introduzione}}

Questo lavoro parte da un'esperienza diretta che ha avuto i natali nel
2004 a Firenze per poi svilupparsi e continuare in Brasile dal 2005 ad
oggi. Nel maggio del 2004, con il Collettivo di Informatica "Ada
Byron"\footnote{Il Collettivo di Informatica "Ada Byron", è un gruppo
  di studenti del Corso di Laurea in Informatica dell’Università degli
  Studi di Firenze. Il sito internet del collettivo è
  \url{http://informada.wordpress.com}.}, organizzammo una serie di
incontri e conferenze sul tema della libertà della conoscenza
applicata a vari ambiti. Parteciparono a "Modelli Liberi"\footnote{Un
  articolo sull'evento è disponibile su
  \url{http://www.apogeonline.com/webzine/2004/04/16/01/200404160103}.},
così si chiamavano gli incontri, relatori di rilievo internazionale,
quali Richard Stallman, Diego Saraiva, Juan Carlos Gentile e, come
rappresentanti del governo brasiliano, Sergio Amadeu e Elaine da
Silva. In seguito, ripresi i contatti con Elaine, e nel 2005, mi
trasferii a Brasilia dove iniziai a lavorare al programma del governo
federale brasiliano per l'universalizzazione dell'accesso a internet,
il GESAC (\emph{Governo Eletrônico Serviço de Atendimento ao
  Cidadão}). Il GESAC era, e tuttora è, il più ambizioso progetto di
lotta al \emph{digital divide} dell'America Latina. Il programma
prevedeva un numero iniziale di 3200\footnote{Ad oggi le connessioni
  sono più di 11000.} connessioni satellitari bidirezionali in tutto il
territorio nazionale, una piattaforma di servizi online e un'equipe
sul territorio. Il Ministero delle Comunicazioni aveva affidato la
direzione del programma a Antonio Albuquerque, tra i fondatori del
Sindacato delle Telecomunicazioni brasiliano, che sostenne e diffuse
l'uso di Software Libero a tutti i livelli, sottolineandone
l'importanza strategica. Il nostro lavoro consisteva nella ricerca e
sperimentazione di soluzioni informatiche applicate ai più svariati
habitat e contesti sociali e culturali. Ci trovavamo continuamente a
contatto con realtà estremamente diverse. Il GESAC serve infatti
scuole, associazioni, caserme, comunità rurali, riserve indigene,
quilombo\footnote{I Quilombo sono comunità fondate da africani
  deportati e afro-discendenti che resistettero e si rivoltarono alla
  schiavitù perpetrata dai colonizzatori portoghesi e europei in tutta
  l'America Latina. Molte delle comunità sono arrivate ai giorni
  nostri e, solo in Brasile, se ne contano circa cinquemila. La
  costituzione brasiliana sancisce il diritto alla terra per i
  Quilombo, anche in segno di riparazione storica. I Quilombo sono per
  la maggior parte localizzati in area rurale e spesso di difficile
  accesso.}, comunità di pescatori, ecc. Nella quasi totalità dei casi
erano persone alla prima esperienza informatica. L'approccio scelto
prediligeva il dialogo e il confronto aperto dove, ad esempio, al
posto di una lezione frontale, si ricorreva ad una \emph{roda de
  conversas}, tutti seduti in cerchio, pratica tra l'altro molto
popolare in Brasile. All'equipe del GESAC furono invitati a lavorare
persone con esperienza nell'ambito sociale, tra cui ad esempio
attivisti di Indymedia\footnote{``Indymedia e' un network di media
  gestiti collettivamente per una narrazione radicale, obiettiva e
  appassionata della verità. Ci impegniamo con amore e ispirazione per
  tutte quelle persone che lavorano per un mondo migliore, a dispetto
  delle distorsioni dei media che con riluttanza si impegnano a
  raccontare gli sforzi dell'umanità libera.'', tratto da
  \url{http://www.indymedia.org/it/static/about.shtml}.} e
Intervozes\footnote{``Intervozes è un'organizzazione che lavora per
  rendere effettivo il diritto umano alla comunicazione in Brasile.
  Per Intervozes, il diritto alla comunicazione è indissociabile dal
  pieno esercizio di cittadinanza e democrazia. Una società può essere
  definita democratica solo quando le diverse voci, opinioni e culture
  che la compongono hanno spazio per manifestarsi.'', tradotto da
  \url{http://www.intervozes.org.br/o-intervozes}.}. Il dialogo aperto
e paritario, a partire dalle specificità del contesto, apriva nuovi
spazi e punti di vista per la rivoluzione digitale, rimettendo al
centro della discussione il fine oltre che il mezzo. Non solo non
portavamo soluzioni preconfezionate, ma spesso gli strumenti e le
pratiche dovevano essere ridiscusse e possibilmente adattate a nuove
necessità.

A questa esperienza, seguì la collaborazione con alcune delle comunità
che erano servite dal GESAC, in particolare con la \emph{Casa de
  Cultura Tainã}\footnote{La \emph{Casa de Cultura Tainã} è un'entità
  sociale e culturale senza fini di lucro fondata nel 1989 da abitanti
  della periferia di Campinas. Obbiettivo dell'entità è rendere
  possibile l'accesso all'informazione fortificando la pratica di
  cittadinanza e la formazione dell'identità culturale, al fine di
  contribuire nella crescita di individui coscienti e attuanti nella
  comunità. L'indirizzo del sito internet è
  \url{http://www.taina.org.br}.} che proprio in quegli anni poneva le
basi del primo nucleo della Rete Mocambos (RM) (vedi
\ref{sec:ReteMocambos}). 

\section{Neutralità tecnologica}
Questa tesi ricerca una soluzione tecnologica per facilitare la
comunicazione tra comunità quilombola. Tali comunità
afro-discendenti condividono tra loro molti aspetti sociali,
culturali, storici e geografici che le differenziano dalle realtà per
la quale, e dalla quale, sono sviluppati normalmente i mezzi di
comunicazione digitale. La neutralità del mezzo viene troppo spesso
data per scontata e risulta difficile percepire quanto questo in
realtà influisca e condizioni le nostre azioni e i nostri
obiettivi. Siamo portati a pensare al mezzo come neutrale, ma se
prendiamo ad esempio i principali e più diffusi mezzi di
comunicazione, le lingue, possiamo intuire come queste non siano
interscambiabili essendo l'espressione delle culture e delle società
che le usano e le vivono. Nell'era digitale è importante considerare
il peso e l'influenza dei nuovi linguaggi e mezzi di comunicazione.

\section{Internet, autonomia e libertà tecnologica}
Negli ultimi anni la diffusione della banda larga, ma sopratutto
strategie come quella adottata da Google, hanno trasformato il
concetto stesso di internet che da rete globale di reti eterogenee,
diventa principalmente una rete per la globalizzazione di servizi
fortemente centralizzati e uniformati. Fino a pochi anni fa era
normale per un'organizzazione provvedere alla gestione, mantenimento e
a volte allo sviluppo di servizi oltre che della infrastruttura
tecnologica per i propri utenti, a partire dalle basi di dati fino ai
servizi per la messaggistica. In questi contesti sono nati e si sono
sviluppate buona parte delle tecnologie di comunicazione oggi
disponibili ma che vengono in parte accantonate per scelte sopratutto
di mercato. Internet è nata dal confronto aperto tra i responsabili di
diverse reti, uniti dalla volontà di connettere le loro differenti
realtà. La nascita di nuovi servizi, per queste reti eterogenee, era
basata sulla discussione e il confronto. Il primo uso documentato del
termine "internet", seguendo questa pratica, fa la sua comparsa
proprio in un RFC \citep{RFC675}. I Request for comments (RFC), sono
dei documenti, normalmente scritti in un linguaggio semplice e
informale, mirati alla diffusione e discussione di nuove tecnologie
nell'ambiente delle telecomunicazioni. Sono anche la principale base
per la definizione di standard e protocolli. Attualmente gli RFC sono
il canale ufficiale di pubblicazione del \emph{The Internet
  Engineering Task Force (IETF)}, del \emph{The Internet Architecture
  Board (IAB)}, tra le altre, e, in generale, della comunità mondiale
dei ricercatori del campo delle reti di comunicazione.

Internet quindi, rete di reti eterogenee, nata dalla pratica della
discussione, del confronto, della ``richiesta di commenti'', sta
ultimamente vivendo l'era dei \emph{Terms of Service (ToS)} su
infrastrutture sempre più omologate e centralizzate, le cosiddette
``nuvole'', o, più comunemente e commercialmente,
\emph{cloud}\footnote{``In informatica, con il termine inglese cloud
  computing si indica un insieme di tecnologie che permettono,
  tipicamente sotto forma di un servizio offerto da un provider al
  cliente, di memorizzare/archiviare e/o elaborare dati (tramite CPU o
  software) grazie all'utilizzo di risorse hardware/software
  distribuite e virtualizzate in Rete.  La correttezza nell'uso del
  termine è contestata da molti esperti: se queste tecnologie sono
  viste da alcuni analisti come una maggiore evoluzione tecnologica
  offerta dalla rete Internet, da altri, come Richard Stallman, sono
  invece considerate una trappola di marketing.'', tratto da
  Wikipedia: \url{http://it.wikipedia.org/wiki/Cloud_computing}.}. Ai
protocolli standard si sostituiscono API proprietarie e suscettibili
di alterazioni continue e unilaterali. Un impresa offre servizi
tramite API, decidendo i termini con cui gli utenti, o altre imprese,
possono accedere e utilizzare questi servizi.

Dal sito di Google:
\begin{quote}
  ``\emph{Google has been pushing the technological bounds of cloud
  computing for more than ten years. Today, feedback and usage
  statistics from hundreds of millions of users in the real world help
  us bring stress-tested innovation to business customers at an
  unprecedented pace. From our consumer user base, we quickly learn
  which new features would be useful in the business context, refine
  those features, and make them available to Google Apps customers
  with minimal delay.}''\footnote{``Google ha spinto i limiti
    tecnologici del cloud computing da più di dieci anni. Oggi, i
    feedback e le statistiche di uso su centinaia di milioni di utenti
    reali nel mondo ci aiutano a portare innovazioni ben testate ai
    nostri clienti business ad un ritmo senza precedenti. Dalla nostra
    base di utenti, apprendiamo velocemente quali caratteristiche
    possono essere utili per il mercato, migliorandole e rendendole
    disponibili agli utenti di Google Apps con un ritardo minimo'',
    Tratto da
    \url{http://www.google.com/apps/intl/en/business/cloud.html}.}
\end{quote}

Come dichiarato inoltre, ``il Cloud computing è nel DNA di Google''
che vanta già 350 milioni di utenti attivi nella sua
nuvola.\footnote{Dichiarazione rilasciata sul ``earning call'' del 19
  gennaio 2012 vedi:
  \url{http://thenextweb.com/google/2012/01/19/gmail-closes-in-on-hotmail-with-350-mm-active-users/}.}.

Secondo la società di consulenza Gartner, nel 2016, tutte le compagnie
contemplate dal \emph{Forbes Global 2000} faranno uso di soluzioni
\emph{cloud} \citep{EY2011}.

In sostanza un tempo lo sviluppo seguiva un modello \emph{bottom up}, per
cui le maestranze informatiche sviluppavano sistemi ad hoc per le
esigenze locali, per poi in seguito aprire una discussione in rete,
con i loro corrispettivi, per definire dei protocolli standard e
mettere in comunicazione il tutto.

Oggi si passa ad un modello di sviluppo \emph{top down}, per cui nuovi
servizi vengono lanciati basandosi su indagini di mercato e test su
campioni di utenti. Poche imprese fanno da padrone nel tracciare lo
sviluppo della rete che assume orizzonti sempre più determinati dal
mercato. Inoltre il mercato di riferimento, e il campione scelto, sono
per lo più legati al contesto economico, sociale e culturale
nordamericano e europeo. 

Pur essendo un analisi di interesse filosofico/antropologico diventa
territorio di frontiera con l'informatica studiare la cultura digitale
e gli effetti che le tecnologie digitali possono avere sulle culture
di paesi e popoli diversi. Una analisi strutturata e analitica ci è
stata lasciata da Vilém Flusser, recentemente riscoperto e considerato
tra i primi filosofi della comunicazione dei nostri tempi, di cui
riporto due pensieri molto attuali, anche se pubblicati già nel 1983
nel libro \emph{F{\"u}r eine Philosophie der Fotografie}:

\begin{quote}
  ``\emph{Both those taking snaps and documentary photographers, however,
  have not understood 'information.' What they produce are camera
  memories, not information, and the better they do it, the more they
  prove the victory of the camera over the human
  being.}''\footnote{``Sia chi si limita a scattare foto, sia i fotografi
documentaristi, non hanno compreso l' 'informazione'. Quello che
producono sono memorie fotografiche, non informazione, e meglio lo
fanno, più confermano la vittoria della macchina fotografica
sull'essere umano''}, \citet{flusser1983philosophie}.
\end{quote}

e 

\begin{quote}
  ``\emph{Our thoughts, feelings, desires and actions are being
    robotized; 'life' is coming to mean feeding apparatuses and being
    fed by them. In short: Everything is becoming absurd. So where is
    there room for human freedom?}''\footnote{``I nostri pensieri,
    sentimenti, desideri e azioni vengono robotizzati; la ``vita''
    comincia a significare nutrire apparati e essere nutriti da
    essi. In breve: tutto sta diventando assurdo. Dov'è dunque lo
    spazio per la libertà umana?''}, \citet{flusser1983philosophie}.
\end{quote}

Un approccio diametralmente opposto alla centralizzazione della rete
sono le tecnologie Peer To Peer (P2P)\footnote{Il principio base del
  P2P prevede una infrastruttura di comunicazione in cui i nodi
  comunicano direttamente tra di loro senza nodi centrali
  preconfigurati. Esistono anche protocolli P2P che prevedono dei
  super-nodi, che in parte riprendono il modello client/server,
  normalmente usati per ottimizzare le prestazioni e risolvere
  problemi di attraversamento di reti mascherate. In generale i
  sistemi P2P sono auto-configuranti o necessitano di una conoscenza
  minima dell'architettura della rete sottostante, dato che si basano
  su protocolli per l'instradamento automatico dei pacchetti. Queste
  tecnologie funzionano bene quando un numero sufficiente di nodi è
  attivo e partecipa al funzionamento della rete. Un esempio di
  protocollo P2P molto diffuso è bittorrent, con cui è possibile
  trasferire dati ad alte velocità quando i contenuti richiesti sono
  presenti su nodi con banda a disposizione.}.

Nonostante le tecnologie P2P siano un alternativa disponibile e già
implementata a vari livelli, presentano alcune caratteristiche per cui
non possono essere ampiamente applicabili al contesto della Rete
Mocambos. Le connessioni tra le comunità della RM sono tutte via
satellite, con banda molto limitata. I protocolli P2P, gravando sulla
banda di molti nodi, per operazioni effettuate da un singolo nodo,
penalizzerebbero proprio la risorsa più scarsa. 

Le necessità di una rete, come la Rete Mocambos, sono di vario tipo e
gestite da entità differenti sebbene federate. Nel prossimo capitolo
vedremo un tipo di rete federata per la RM.

 % Introduzione

% Capitulo 2

\chapter{Redes federadas eventualmente conectadas}
\label{Capitulo2}
\lhead{Cap\'itulo 2. \emph{Redes federadas eventualmente conectadas}}


Com rede federada pode se entender um universo muito amplo de
situações. Neste trabalho se considera principalmente uma rede
federada que tenha as seguintes caraterísticas:

\begin{itemize}
  \item administração descentralizada
  \item acesso a gestão lógica (e física) da rede
  \item conhecimentos técnicos locais das tecnologias usadas na rede
  \item serviços internos a rede federada
  \item interações entre redes locais inteligentes
\end{itemize} 

Uma rede federada, então, entendida como um conjunto de soluções
tecnicamente viáveis e adaptáveis a usos, práticas e contextos
diferentes, que permita uma gestão flexível da rede, com sub-redes
heterogêneas, aproveitando diversas tecnologias, como por exemplo P2P
onde necessário. Uma rede federada se adapta particularmente num
contexto onde já existe uma estrutura organizacional que pode se
espelhar na estrutura da rede e que pode acompanhar sua gestão. 

\section{Redes federadas eventualmente conectadas}
A introdução sobre o contexto sugere o âmbito de estudo mas necessita
uma restrição maior. Para ``redes federadas eventualmente conectadas''
consideramos, uma rede federada baseada em conexões não sempre
disponíveis, como as conexões por satélite, com a exigência de manter
os serviços federados ativos, mesmo em ausência de comunicação. Os
serviços federados são, ainda, otimizados para a resiliência do
sistema e a redução do trafego de rede externo, através de estrategias
de replicação, sincronização e memorização dos dados na infraestrutura
logico/física local.

\section{A Rede Mocambos}   
\label{sec:RedeMocambos}

\begin{quote}
  ``\emph{É uma rede solidária de comunidades, no qual o objetivo
    principal é compartilhar ideias e oferecer apoio recíproco.}''
  \ldots ``\emph{A tecnologia é uma frente de trabalho da Rede
    Mocambos, sendo ao mesmo tempo ideia e meio para transferir
    ideias.}'' \citep{RMSobre}.
\end{quote}

A Rede Mocambos (RM), atualmente envolve diretamente mais de duzentas
comunidades quilombolas, coletivos, aldeias indígenas, pontos de
cultura e terreiros (ver figura \ref{fig:MappaRedeMocambos}). Existem
dois termos de cooperação\footnote{Os termos de cooperação foram
  assinados, como Rede Mocambos, mas formalmente pela Casa de Cultura
  Tainã.} entre Rede Mocambos e o Ministério das Comunicações,
precisamente com os programas GESAC e Telecentros.BR.

\begin{figure}[htbp]
  \centering
  \includegraphics[width=\textwidth]{./Figure/MappaRedeMocambos.pdf}
  \rule{35em}{0.5pt}
  \caption[Mapa das comunidades da Rede Mocambos, retirado deste
  \url{http://mapa.mocambos.net}]{Mapa das comunidades da Rede Mocambos, retirado desde
  \url{http://mapa.mocambos.net}}
  \label{fig:MappaRedeMocambos}
\end{figure}

O GESAC garante conectividade por satélite a todas as comunidades, com
banda limitada devido a tipologia de tecnologia, que devido as
distancias representa a única alternativa, pelo menos no curto e médio
prazo. Através do programa Telecentros.BR estão sendo instalados
telecentros, salas equipadas com computadores para acesso público, e
garantidas bolsas para monitores, para a gestão desses espaços. É
exatamente em contextos tão específicos que nasce a necessidade de
adaptar a tecnologia as exigências locais, também pelas limitações
técnicas impostas. A escassez de banda leva a reconsiderar a rede não
somente como o meio de conexão para os grandes \emph{data centers}. A
rede pode, e neste caso deve, ser estruturada no território com
logicas de desenvolvimento e gestão local determinadas pelas
comunidades. Neste sentido é fundamental a formação e o acesso as
tecnologias. A Casa de Cultura Tainã, núcleo fundador da Rede
Mocambos, e entre as primeiras realidades populares que perceberam a
necessidade do Software Livre, como expressão de liberdade de poder
criar as próprias ferramentas tecnológicas digitais. A exclusão
digital não diz respeito somente ao acesso ao mundo digital mas também
da impossibilidade de participar a sua criação e desenvolvimento.



% \begin{figure}[htbp]
%   \centering
%   \includegraphics[width=\textwidth]{./Figure/taina_oficina.pdf}
%   \rule{35em}{0.5pt}
%   \caption[Laboratorio di grafica 3D con Blender alla \emph{Casa de
%     Cultura Tainã}]{Laboratorio di grafica 3D con Blender alla
%     \emph{Casa de Cultura Tainã}}
%   \label{fig:oficinaBlender}
% \end{figure}

\begin{figure}[htbp]
  \centering
  \includegraphics[width=\textwidth]{./Figure/FotoTesi.pdf}
  \rule{35em}{0.5pt}
  \caption[Fotos da Rede Mocambos]{Fotos da Rede Mocambos}
  \label{fig:FotoRM}
\end{figure}

Para a RM é importante, por vários aspectos, poder construir e gerir
os meios de comunicação e adaptá-los a seu contexto. No Brasil são
muitas as comunidades indígenas, quilombolas e tradicionais que
conhecem bem as potencialidades das tecnologias digitais e estão, cada
vez mais, dominando-as. Isso não teria sido possível sem a existência
e ampla difusão de Software Livre. Tecnologias digitais sob forma de
produtos comerciais seriam o enésimo passo para uma dependência
econômica e cultural. Esta consciência esta atras de escolhas bem
ponderadas. Alguns anos atras, em Brasília, numa reunião do programa
Luz Para Todos, programa federal para levar a eletricidade a população
da área rural, um cacique disse que, mesmo querendo a energia
elétrica, esta deveria ser limitada aos espaços comunitários e não
residenciais. Não é difícil entender o porque de tal condição. Além
dos aspectos culturais, ter um contador e uma conta para cada casa
significaria introduzir custos em dinheiro, que não são compatíveis
com a economia deles.

\section{Tecnologias}
Antes de mergulhar nos requisitos específicos, e nas escolhas feitas,
pode ser útil uma breve descrição das ferramentas tecnológicas que de
várias formas foram usadas para estruturar o protótipo, para a Rede
Mocambos, de rede federada eventualmente conectada, em particular para
os seus serviços de base, como a identificação, autenticação e
comunicação.

\subsection{LDAP}
\emph{Lightweight Directory Access Protocol (LDAP)}\footnote{Protocolo
  Leve de Acesso a Diretório.} é um conjunto de protocolos abertos
para acessar as informações mantidas centralmente através de uma
rede. LDAP organiza as informações através de uma hierarquia a árvore
chamada \emph{Directory Information Tree (DIT)}\footnote{Arvore do
  diretório de informações}. LDAP é um sistema cliente/servidor. O
servidor pode usar uma variedade de banco de dados para armazenar um
DIT, normalmente otimizados para operações de leitura. Quando uma
aplicação cliente se conecta a um servidor LDAP, pode consultar o
diretório ou tentar alterá-lo. No caso de uma consulta, o servidor
pode responder de maneira local, ou pode encaminhar o pedido para um
servidor LDAP em condição de responder. Se a aplicação cliente está
tentando modificar informações do diretório LDAP, o servidor verifica
se o usuário possui permissão de efetuar a mudança, para em seguida
adicionar e atualizar as informações. LDAP suporta a delegação de
parte do DIT para servidores específicos, a replicação só em leitura e
a replicação em leitura/escritura (chamada \emph{multi-master}). LDAP
é um protocolo solido, muito comum e suportado, e desde muito tempo é
o padrão de fato para gerir base de dados de usuários. OpenLDAP é uma
implementação aberta e livre do protocolo LDAP, que inclui cliente,
servidor e uma série de ferramentas para facilitar a administração.
  
\subsection{XMPP}
Extensible Messaging and Presence Protocol (XMPP)\footnote{Protocolo
  extensível para mensagens e presencia.} é um conjunto de protocolos
abertos para as mensagens e a presença em rede baseado no XML. XMPP é
um sistema cliente/servidor. As especificações para a comunicação
entre servidores permitem que os usuários de um servidor interajam de
maneira transparente com usuários de outros servidores federados. A
\emph{XMPP Standard Foundation (XSF)} coordena o desenvolvimento das
extensões do padrão por meio das \emph{XMPP Extensions Protocols
  (XEPs)}, que até hoje são já 311. XMPP e as XEPs constituem um
ambiente flexível e completo para o desenvolvimento de serviços
federados. Estes protocolos são já usáveis graças as muitas
implementações de servidores, clientes e bibliotecas livres. A
história do XMPP, uma vez conhecido como Jabber, também é
interessante. Jabber foi inicialmente desenvolvido por Jeremie Miller
na sua fazenda no Iowa. É um exemplo concreto de como a pesquisa e o
desenvolvimento de tecnologias da comunicação, fora de ambientes
acadêmicos e empresariais, além de possível pode ser
revolucionária. De fato, hoje, XMPP é a tecnologia mais usada para
mensagens instantâneas também pelo grandes atores da \emph{new
  economy}.

Para as necessidade específicas de uma rede é possível então estender
e customizar as funcionalidades do próprio servidor XMPP a ao mesmo
tempo usufruir dos serviços base, mensagens e presencia, implementados
pelos servidores já existentes na rede. 

\subsection{OpenID}
OpenID é um sistema de identificação descentralizada no qual a própria
identidade é uma URL que pode ser verificada por qualquer serviço que
suporte o protocolo. É um protocolo aberto e são disponíveis várias
implementações livres. Além disso o protocolo foi adotada pelos
principais provedores de serviços web. Com OpenID é possível usar a
mesma identidade em mais serviços e é uma ótima base para um sistema
\emph{Single Sign On (SSO)}. O protocolo utiliza HTTP e \emph{Cookies}
para manter uma sessão ativa. No primeiro tentativo de autenticação em
um serviço compatível com OpenID, o usuário é redirecionado para o
próprio provedor OpenID para efetuar o acesso e confirmar a
autorização a proceder com o serviço inicial. Por todo o período da
sessão, é possível acessar serviços OpenID sem reinserir as
credenciais.

\begin{figure}[htbp]
  \centering
  \includegraphics[width=0.6\textwidth]{./Figure/OpenID_Scenario.pdf}
  \rule{35em}{0.5pt}
  \caption[Diagrama de autenticação de OpenID]{Diagrama de autenticação de OpenID}
  \label{fig:OpenID}
\end{figure}

\subsection{OAuth}
OAuth é um protocolo aberto para autorização de serviços através
API. Por exemplo, permite um usuário autorizar o acesso a
informações específicas armazenadas num site, chamado \emph{service
  provider}, para outro site, chamado \emph{consumer}, sem necessidade
de compartilhar a sua identidade. É uma maneira de publicar e
interagir com dados protegidos. Existem muitos outros protocolos e API
parecidos como Google AuthSub, AOL OpenAuth, Yahoo
BBAuth, Upcoming API, Flickr API, Amazon Web Services e cada um
fornece maneiras proprietárias para troca de credenciais e para acesso
através de \emph{tokens}. OAuth é uma padronização aberta das práticas
mais comuns. Além disso foi pensado para suportar vários tipos de
aplicações, não somente para serviços web.

\subsection{Shibboleth}
Shibboleth é uma arquitetura e uma implementação aberta para
autenticação e autorização de identidades federadas baseada no
\emph{Security Assertion Markup Language (SAML)}\footnote{Linguagem de
marcação para asserções de segurança.}. As identidades federadas
permitem que as informações de um usuário sob um certo domínio possam
ser compartilhadas com um outro domínio federado. Isso permite o SSO
através de mais domínios sem a troca de nomes de usuários e senhas. Os
IdP mantêm as informações do usuário enquanto os SP fazem uso dessas
informações para o acesso seguro aos conteúdos. 

\subsection{Kerberos}
Kerberos é um protocolo aberto para autenticação forte em redes de
computadores. É um protocolo cliente/servidor e permite a autenticação
reciproca, ou seja ambos verificam suas identidades. Kerberos é
baseado no protocolo de Needham-Schroeder a chaves simétricas e prevê
uma entidade terceira de confiança chamada \emph{Key Distribution
  Center (KDC)}.
 
\subsection{Django}\label{Django}
Django é um framework, escrito em Python, para o desenvolvimento
rápido de aplicações web, também conhecido como ``\emph{The web
  framework for perfectionists with deadlines}''\footnote{``O
  framework web para perfeccionistas com prazos''.}, pois foi criado
com o objetivo de respeitar o máximo possível os princípios
DRY\footnote{``Don't Repeat Yourself, não se repita, (DRY, também
  conhecido como ``Single Point of Truth'', ponto único de verdade) é
  um principio segundo o qual a informação não tem que ser repetida e
  redundante e não tem que expressar o mesmo conceito mais de uma
  vez.'', traduzido do Wikipedia:
  \url{http://it.wikipedia.org/wiki/Don\%27t_Repeat_Yourself}.}, e
então oferece todas as ferramentas para escrever código limpo de
maneira pragmática e eficaz.

As caraterísticas do Django são:
\begin{itemize}
\item é um framework \emph{Model, View, Template}, (MVT),
  correspondente ao difundido \emph{pattern Model, View, Controller},
  (MVC). O \emph{Template} do Django corresponde ao \emph{view} de
  qualquer framework MVC, enquanto a \emph{View} corresponde ao
  \emph{controller} (mesmo se as funcionalidades do \emph{controller},
  no caso do Django, não são limitadas a componente \emph{View}).
\item permite modelar os dados diretamente em python e o
  \emph{Object Relational Mapper} (ORM), se ocupa de transformá-los em
  código SQL. Também não precisa escrever \emph{query} diretamente em
  SQL; é só usar o ORM para obter os resultados diretamente sob forma
  de objetos python. O ORM de Django suporta PostgreSQL, MySQL, sqlite,
  Microsoft SQL Server e Oracle.
\item possui uma biblioteca para \emph{form} realmente poderosa e
  expressiva que permite realizar logicas complexas com poucas linhas
  de código, deixando o máximo controle ao desenvolvedor para todas as
  situações. 
\item é um framework ``\emph{batteries included}'', ou seja já vêm com
  algumas aplicações, chamadas \emph{apps}, para as funcionalidade
  mais comuns, que reduzem o tempo de desenvolvimento, por exemplo um
  sistema de autenticação, uma interface de administração, um
  framework para gerar \emph{sitemaps XML} e muito mais. Além disso a
  comunidade é bem ativa e existem muitas \emph{apps} adaptáveis e
  reusáveis. 
\item é entre os melhores framework quanto a documentação
  oficial. Além de alguns tutoriais, uteis para aprender as bases,
  todas as funcionalidades e características do framework são
  bem documentadas.
\end{itemize}


\subsection{Git}\label{sec:GIT}
Git é um sistema multiplataforma para o controle de versão
distribuído, projetado para ser rápido e usável mesmo em grande
projetos. 

As características principais incluem:
\begin{itemize}
\item é totalmente distribuído e cada clone de um repositório contem o
  histórico inteiro das versões e no qual podem ser efetuadas
  operações independentemente de conexões de rede o de servidores
  centrais. As mudanças podem ser copiadas entre um clone e o outro e
  são mantidos em \emph{branch} (ramos) diferentes, facilitando as
  operações de \emph{merge} (fusão). Os repositórios são facilmente
  acessíveis através do eficiente protocolo do Git, que além de
  suportar HTTP, pode funcionar junto com SSH, para obter conexões
  seguras e um sistema de autenticação solido e bem comum.
\item suporta o \emph{branching} (ramificação), e o \emph{merging}
  (fusão), de maneira rápida e conveniente, incluindo uma serie de
  ferramentas para visualizar e navegar o histórico não linear das
  versões.
\item é muito rápido e escala mesmo em projetos muito grandes e com
  muitas mudanças, graças a um eficiente sistema de empacotamento e
  memorização do histórico (é considerado o mais eficiente entre os
  sistemas atualmente disponíveis).
\item associa um nome de versão, para cada \emph{commit}, que é função
  do histórico inteiro, por isso, uma vez publicada uma versão, não é
  possível alterar as velhas sem ser notado. As versões podem também
  ser etiquetadas e assinadas digitalmente com GPG.
\end{itemize}

Git é um sistema completo que, em bom estilo Unix, é organizado em
programas e comandos independentes, pensados para ser facilmente
usáveis, seja automaticamente através de \emph{scripting} seja de
maneira interativa pelo usuário final. Git é, então, uma base solida
para o desenvolvimento de aplicações orientadas a sincronização, a
portabilidade e a gestão autônoma e descentralizada. 


 % Reti Federate  

% Cápitulo 3

\chapter{Un'architettura per la Rete Mocambos}
\label{Capitolo3}
\lhead{C\'apitulo 3. \emph{Uma arquitetura para Rede Mocambos}}

\section{Especificação dos requisitos}
A Rede Mocambos atualmente envolve cerca de 200 comunidades
localizadas em todo o território nacional brasileiro e algumas
comunidades na Africa e na Europa. A conectividade no Brasil é por
satélite, garantida pelo GESAC que disponibiliza para cada comunidade
uma \emph{Very Small Aperture Terminal} (VSAT) com banda de 512 kbit/s
em download e 128 kbit/s em upload. A topologia da rede é a estrela
então todos os nós comunicam por satélite concentrando o trafego num
\emph{hub} terrestre, onde a rede via satélite é interligada a
Internet. Cada comunidade tem uma sala com 10 computadores com acesso
publico, recém instalados (ou sendo instalados) pelo
Telecentros.BR\footnote{``O Programa Nacional de Apoio à Inclusão
  Digital nas Comunidades – Telecentros.BR é uma ação do Governo
  Federal de apoio à implantação de novos espaços públicos e
  comunitários de inclusão digital e o fortalecimento dos que já estão
  em funcionamento em todo o território. São disponibilizados
  equipamentos de informática e mobiliário necessários ao
  funcionamento dos telecentros, serviços de conexão em banda larga à
  internet, assim como a formação e bolsas de auxílio financeiro para
  monitores atuarem como agentes de inclusão digital. Esses monitores
  bolsistas participam de um curso de formação e atendem as
  comunidades dos telecentros.'', retirado de
  \url{http://www.inclusaodigital.gov.br/telecentros}.}, outro
programa do governo federal. A população de casa comunidade varia
desde as centenas as milhares de pessoas. A maioria das comunidade se
encontra em área rural, geralmente de difícil acesso e sem outros
meios de comunicação. Além dos espaços comunitários, normalmente
concentrados na zona central, a população é dividida em pequenos
núcleos familiares espalhados no território e as vezes muito distantes
um dos outros.

Vejamos agora alguns requisitos essenciais para essa arquitetura de
rede federada.

\subsection{Identidade de rede}
Os usuários das comunidades tem que ter identidade digitais com as
quais acessar e usar os serviços existentes. Além de acessar
localmente, é importante ter acesso a alguns serviços também fora da
própria comunidade. Por exemplo, no caso uma pessoa se encontra na
cidade, ela deve poder acessar, através da internet, aos serviços de
base, como o e-mail, o aos portais e serviços web da RM. É necessária
então uma identidade digital de rede univoca. 

\subsection{Autenticação descentralizada}
\label{sec:AutDec}
Um requisito essencial para cada comunidade é ter autonomia em
ausência de conexão internet, e a presença então de um sistema local
de autenticação e autorização. Além de garantir a continuidade do
serviço (em relação a problemas da conexão por satélite), é um
requisito importante para um bom desempenho de todos os serviços autenticados. 

\subsection{Sincronização}\label{Sincronizazzione}
A troca de informações entre as comunidades é o objetivo primário para
a RM. É então necessário um sistema para a sincronização seletiva dos
conteúdos de interesse geral. Para otimizar o uso da banda, as
operações de sincronização dos dados tem que influenciar o minimo
possível o uso quotidiano da internet, programando essas operações
durante a noite o de qualquer forma quando a conexão por satélite não
ta sendo usada. 

\subsection{Replicabilidade}
Cada comunidade se organiza autonomamente para a gestão da própria
infraestrutura tecnológica e as soluções adotadas tem então que ser
facilmente implementáveis e adaptáveis localmente. As ferramentas
precisam ser então soluções estáveis e possivelmente de uso comum,
também ao fim de encontrar mais facilmente assistência em loco. 

\subsection{Manutenção}
A manutenção do sistema tem que prever seja intervenções a distancia
seja locais. O sistema é baseado em tecnologias \emph{standard} e
abertas para garantir o acesso aos dados mesmo em caso de problemas.

\subsection{Desenvolvimento}
Cada comunidade tem características e necessidades próprias que levam
a serviços diferenciados. A arquitetura geral da RM tem que ser uma
base estável em cima da qual poder desenvolver sem demais restrições.
As tecnologias e as linguagens adotadas tem que facilitar a interação,
a formação e o reuso dos conhecimentos. 


======================================================================



\section{Strumenti e pratiche per lo sviluppo}
Data la natura della RM, e in particolare la necessità di autonomia
tecnologica, la formazione è fondamentale per cui è importante l'uso
di strumenti che facilitino la documentazione e lo sviluppo
collettivo. La RM utilizza già da anni un wiki\footnote{``Una Wiki è
  una pagina (o comunque una collezione di documenti ipertestuali) che
  viene aggiornata dai suoi utilizzatori e i cui contenuti sono
  sviluppati in collaborazione da tutti coloro che vi hanno
  accesso. La modifica dei contenuti è aperta, nel senso che il testo
  può essere modificato da tutti gli utenti (a volte soltanto se
  registrati, altre volte anche anonimi) contribuendo non solo per
  aggiunte come accade solitamente nei forum, ma anche cambiando e
  cancellando ciò che hanno scritto gli autori precedenti.  Ogni
  modifica è registrata in una cronologia che permette in caso di
  necessità di riportare il testo alla versione precedente; lo scopo è
  quello di condividere, scambiare, immagazzinare e ottimizzare la
  conoscenza in modo collaborativo. Il termine wiki indica anche il
  software collaborativo utilizzato per creare il sito web e il
  server.'', tratto da \url{http://it.wikipedia.org/wiki/Wiki}.},
disponibile all'indirizzo \url{http://wiki.mocambos.net}, dove è
disponibile, in lingua portoghese, una parte della documentazione sul
codice sviluppato per questo lavoro. Per il \emph{versionamento} del
codice del lavoro svolto si è scelto di usare il sistema decentrato
GIT (vedi \ref{sec:GIT}) e la piattaforma
GITHUB\footnote{\href{http://github.com}{GITHUB} è una rete sociale
  basata sul sistema GIT \ref{sec:GIT}.}. Il codice del prototipo è
disponibile all'indirizzo \url{https://github.com/RedeMocambos}

\subsection{Sistema operativo}
La scelta di un sistema operativo comune facilita la documentazione,
l'automazione e l'assistenza a distanza. Molte comunità utilizzano già
distribuzioni GNU/Linux basate su Debian\footnote{``Il Progetto Debian
  è una associazione di persone che ha come scopo comune la creazione
  di un sistema operativo libero. Il sistema operativo che abbiamo
  creato si chiama Debian GNU/Linux, o semplicemente Debian.'', tratto
  da \url{http://www.debian.org/}}. Per il prototipo sono state scelte
l'ultima versione stabile, di Debian, la 6.0, e di
Ubuntu\footnote{``Ubuntu è un sistema operativo GNU/Linux nato nel
  2004, basato su Debian, che si focalizza sull'utente e sulla
  facilità di utilizzo. Ubuntu è orientato all'utilizzo desktop e pone
  una grande attenzione al supporto hardware. È prevista una nuova
  versione ogni sei mesi.  Finanziato dalla società Canonical Ltd
  (registrata nell'Isola di Man), questo sistema è rilasciato come
  software libero sotto licenza GNU GPL ed è gratuito e liberamente
  modificabile.'', tratto da
  \url{http://it.wikipedia.org/wiki/Ubuntu}.}, la 11.10.

\subsection{Linguaggi di programmazione}
Per il sistema sviluppato si è scelto l'uso del linguaggio di
programmazione Python, un linguaggio molto flessibile e probabilmente
il più versatile per connettere componenti eterogenee. Inoltre alcune
comunità già usano Software Liberi, quali
\emph{GIMP}\footnote{\emph{GNU Image Manipulation Program (GIMP)} è un
  programma per il fotoritocco, montaggio e creazione di
  immagini. Disponibile su \url{http://www.gimp.org/}.},
\emph{Blender}\footnote{\emph{Blender} è un programma per la creazione
  di grafica e ambienti tridimensionali. Disponibile su
  \url{http://www.blender.org/}.},
\emph{Inkscape}\footnote{\emph{Inkscape} è un programma di grafica
  vettoriale conforme allo standard SVG. Disponibile su
  \url{http://inkscape.org}.} che fanno ampio uso di \emph{scripting}
Python per la creazione di filtri e per altre funzionalità avanzate.

\subsection{Virtualizzazione}
Il prototipo è stato sviluppato e testato in un ambiente
virtualizzato, per garantire maggior controllo e verificare la
correttezza del procedimento in tutti i suoi passaggi. Gli
\emph{script} per automatizzare le installazioni, e le procedure passo
passo presenti nella documentazione, sono state eseguite a partire da
un sistema base. Sono state create delle macchine virtuali, con
l'aiuto del programma libero
\emph{VirtualBox}\footnote{\emph{VirtualBox} è un virtualizzatore
  completo per architettura x86 per l'uso in server, computer
  personali e sistemi \emph{embedded}. Dispobibile su
  \url{http://www.virtualbox.org/}.}, simulando dei server
comunitari/locali e il server centrale.


\section{Architettura di base}

\subsection{Gestione delle identità di rete}
Analizzando la specifica dei requisiti e alcune delle tecnologie
esistenti viste nel capitolo precedente, è stato scelto l'uso di LDAP
per gestire le credenziali degli utenti all'interno della RM. Il
prototipo proposto prevede un server centrale su connettività internet
garantita e server locali in replica. È possibile pensare una
configurazione in cui tutti i server della rete sono configurati in
modalità N-Way-Multimaster. Questa configurazione, se da un lato
consente l'aggiornamento in scrittura della base utenti su ognuno dei
server, può generare più facilmente situazioni di inconsistenza dei
dati e problemi di sincronizzazione tra i server LDAP. Per questo
motivo per il prototipo realizzato si è scelto di rilassare il
requisito \ref{sec:AutDec}, ipotizzando un singolo server master per
tutta la rete su cui effettuare le operazioni in scrittura (creazione,
eliminazione e aggiornamento di utenti) e server locali abilitati solo
ad operazioni in lettura. Ogni comunità quindi avrebbe a disposizione
un server LDAP in replica su cui basare le operazioni di
autenticazione e autorizzazione. In mancanza di connessione esterna i
servizi locali rimangono comunque attivi per le utenze attive fino
all'ultima sincronizzazione.


\subsection{Mocambos\_LDAP}\label{MocambosLDAP}
\framebox[\textwidth]{\footnotesize Il codice è disponibile su
\url{https://github.com/RedeMocambos/Mocambos_LDAP}}

Per facilitare l'implementazione dei server LDAP dagli amministratori
locali è stato sviluppato uno script di installazione, che può tornare
utile anche per un amministratore remoto.

Lo script \textit{bash} provvede ad installare i pacchetti necessari e
preparare il server LDAP \emph{slapd} in configurazione master o
replica. Inoltre crea un DIT preconfigurato per la RM con una semplice
struttura (vedi figura \ref{fig:DIT_ReteMocambos}).

\begin{figure}[htbp]
  \centering
  \includegraphics[width=\textwidth]{./Figure/DIT_ReteMocambos-crop.pdf}
  \rule{35em}{0.5pt}
  \caption[DIT di base del server LDAP della RM]{DIT di base del
    server LDAP della RM.}
  \label{fig:DIT_ReteMocambos}
\end{figure}




 % Architettura

% Capitulo 4

\chapter{Um protótipo de serviço federado}
\label{Capitulo4}
\lhead{Cap\'itulo 4. \emph{Um protótipo de serviço federado}}

O uso de tecnologias digitais oferece novas possibilidades para o
ensino e a formação. Muitas comunidades, aferentes a RM, vem
produzindo material pedagógico audiovisual, que com a ajuda da rede,
poderiam contribuir para enriquecer os programas educacionais
públicos, geralmente deficientes e pouco integrados com a cultura
quilombola. O \emph{Núcleo de Pesquisa e Desenvolvimento Digital
  (NPDD)}\footnote{O \emph{Núcleo de Pesquisa e Desenvolvimento
    Digital (NPDD)} da RM pesquisa e desenvolve tecnologias para a
  comunicação, a produção de energias renováveis e sustentáveis, e a
  melhoria das condições de vida em simbiose com o ambiente. Mais
  informações em \url{http://wiki.mocambos.net/wiki/NPDD}.} da RM, com
o projeto \emph{Tambor e Comunicação}\footnote{O projeto \emph{Tambor
    e Comunicação} tenta fortificar a rede de comunicação digital
  seguindo as necessidades das comunidades. Ver
  \url{http://wiki.mocambos.net/wiki/Projeto_Tambor_e_Comunicacao}.},
propus a pesquisa e o desenvolvimento de uma solução para a publicação
e compartilhamento em rede de imagens, áudios e vídeos de interesse
comum e geralmente produzidos nas comunidades.

\section{Sistema de publicação e sincronização de conteúdos multimídia}
O sistema desenvolvido prevê a instalação de um portal no servidor
local das comunidades através do qual é possível visualizar e publicar
conteúdos multimídia aproveitando a rapidez da rede local. O sistema
cuida de memorizar os conteúdos em um acervo multimídia local e
compartilhar, aqueles etiquetados como de interesse comum, com os
servidores das outras comunidade (ver figura
\ref{fig:SchemaServer_ReteMocambos}). O protótipo tenta resolver as
limitações de banda da conexão respeitando a especificação dos
requisitos.

\begin{figure}[htbp]
  \centering
  \includegraphics[width=\textwidth]{./Figure/SchemaServer_ReteMocambos-crop.pdf}
  \rule{35em}{0.5pt}
  \caption[Esquema da infraestrutura da RM]{Esquema da infraestrutura da RM.}
  \label{fig:SchemaServer_ReteMocambos}
\end{figure}

\section{Acervo multimídia}
O acervo multimídia local da comunidade é um repositório
\emph{git-annex} (ver \ref{git-annex}) que vem gerenciado através de
um portal comunitário, seja pela publicação seja pela visualização dos
conteúdos. O acesso direto aos dados é todavia garantido sendo esses
arquivos acessíveis no disco. Também é possível interagir com os dados
através do amplo universo de aplicações e bibliotecas do
\emph{git}. Em especial o protótipo usa a estrutura de metadados do
\emph{git} para manter o controle do usuário que publicou um
conteúdo. A operação de \emph{commit} coincide de fato com a
publicação de um conteúdo, e o \emph{committer} com o usuário que o
publica. Os metadados, em vez, relativos ao conteúdo multimídia em si,
como autor, tipo de licenciamento, data de criação, podem ser
memorizados internamente ao arquivo seguindo os padrões existentes
como o \emph{Dublin Core}\footnote{``O Dublin Core é um sistema de
  metadados que contempla um núcleo de elementos essenciais para a
  descrição de qualquer conteúdo digital acessível via rede.'',
  traduzido de Wikipedia:
  \url{http://it.wikipedia.org/wiki/Dublin_Core}.}.

Um elemento importante para o acervo multimídia são as operações de
sincronização. As ferramentas baseadas no \emph{git} herdam a sua
natureza descentralizada e a capacidade de comunicar de forma
transparente usando vários protocolos. Em particular é interessante a
possibilidade de executar sincronizações com sistemas de armazenamento
massivo, característica essencial na fase de criação de um novo nó,
onde a primeira sincronização via rede poderia levar dias (ver
requisito \ref{sec:Sinc}). As transferências contudo, no caso
do \emph{git-annex}, são executadas através do protocolo
\emph{rsync}\footnote{\emph{rsync} é um Software Livre para a
  transfêrencia rápida e incremental de arquivos disponível no
  \url{http://rsync.samba.org/}.}, que gerencia eventuais
interrupções, evitando retransmissões onerosas. 

\subsection{git-annex}\label{git-annex}
\emph{git-annex}\footnote{\emph{git-annex} é um programa que estende
  as funcionalidades do \emph{git} em gerir arquivos de grande tamanho
disponível no \url{http://git-annex.branchable.com}.} permite a
gestão de arquivos com \emph{git}, sem a necessidade de adicionar os
arquivos dentro \emph{git}. Mesmo se pode parecer paradoxal, é útil
quando se trabalha com arquivos muito grandes que \emph{git}
atualmente não gerencia facilmente por limitações devidas a memoria,
tempo ou espaço no disco.

Mesmo sem manter o histórico das mudanças do conteúdo do arquivo, ter
a possibilidade de gerenciar arquivos com \emph{git}, de movê-los, e
exclui-los, numa árvore de pasta versionada, com uso de
\emph{branches} e de clones distribuídos, são todos bons motivos para
usar \emph{git}. E os arquivos anexos (por isso o nome
\emph{git-annex}) podem coexistir no mesmo repositório \emph{git} com
os arquivos regularmente versionados. 

\emph{git-annex} transforma os arquivos anexos em \emph{link}
simbólicos, que são normalmente versionados por \emph{git}. 

O conteúdo dos arquivos é mantido por \emph{git-annex} em um acervo
chave/valor distribuído que corresponde aos clones de um dado
repositório \emph{git}. Praticamente \emph{git-annex} memoriza o
conteúdo do arquivo em uma subpasta de \verb|.git/annex/|.

A primeira vez que um arquivo é adicionado no \emph{git-annex}, é
calculada uma chave, normalmente fazendo um \emph{hash} do seu
conteúdo. \emph{git-annex} todavia suporta vários \emph{backend} que
podem produzir diferentes tipos de chaves. O arquivo que é adicionado
no \emph{git} nada mais é que um \emph{link} simbólico para a chave
memorizada no \verb|.git/annex/|. Se o conteúdo do arquivo for
modificado, é gerada uma outra chave, e o \emph{link} é alterado. 

O conteúdo do arquivo pode ser transferido de um repositório para
outro por \emph{git-annex}, que além de manter controle de quem mantem
o que, permite criar um mapa das copias disponíveis e impor um número
mínimo de cópias. Essas informações são mantidas em um \emph{branch}
separado, chamado ``\emph{git-annex}'', e as operações de
sincronização, são simplesmente \emph{push} e \emph{pull} entre os
vários clones dos repositórios.

\emph{git-annex} suporta:
\begin{itemize}
\item localização das cópias (\emph{location tracking})
\item download seletivo dos conteúdos 
\item gestão da confiança dos repositórios
\item gestão do número minimo de cópias
\item vários \emph{backend} para as chaves (SHA\footnote{Secure Hash
    Algoritm, (SHA), é um algoritmo usado em sistemas chave/valor onde
    as chaves são calculadas através de uma função criptográfica dos
    valores.}, WORM\footnote{O algoritmo WORM identifica os arquivos
    em base ao nome, dimensão e data de alteração.})
\item vários \emph{backend} para os conteúdos/valores
  (BUP\footnote{BUP é um sistema para \emph{backup} a alta eficiência
    disponível no: \url{https://github.com/apenwarr/bup}.}, rsync,
  web, S3\footnote{Amazon Simple Storage Service, (S3) é uma
    infraestrutura para a memorização dos dados totalmente redundante,
    disponível no: \url{aws.amazon.com/}.})
\end{itemize}

\section{Portal Comunitário}
O portal local tem que oferecer acesso aos principais serviços locais
da comunidade. Para o desenvolvimento foi escolhido o uso de um
framework baseado no Python, \emph{Django} (ver \ref{Django}), que
possibilita uma integração flexível e avançada com outros sistemas
graças as numerosas bibliotecas disponíveis como, por exemplo, para a
autenticação LDAP.

Para a instalação do framework e do modelo de base do protótipo de
portal comunitário foi criado um \emph{script} enquanto, para gerir o
acervo multimídia \emph{git-annex} foram desenvolvidos duas aplicações
para \emph{Django}, que definem o modelo dos dados e cuidam de
adicionar os conteúdos no repositório executando as operações de
\emph{commit}, \emph{push} e \emph{pull}.

\begin{figure}[htbp]
  \centering
  \includegraphics[width=\textwidth]{./Figure/UML_Schema_Django-crop.pdf}
  \rule{35em}{0.5pt}
  \caption[Esquema UML das aplicações Django]{Esquema UML das aplicações Django.}
  \label{fig:SchemaUMLDjango}
\end{figure}

\subsection{Mocambos\_Portal\_Local}
\framebox[\textwidth]{\footnotesize O código esta disponível no
  \url{https://github.com/RedeMocambos/Mocambos_Portal_Local}}

O \emph{script bash}, \verb|script/install-django-env.sh|, instala os
pacotes de sistema necessários, e também cria o ambiente virtual,
usando o programa \emph{virtualenv}\footnote{\emph{Virtualenv} è um
  Software Livre para criar ambientes Python isolados disponível no
  \url{http://www.virtualenv.org}.}, que permite instalar facilmente
versões específicas de bibliotecas e do interprete Pyhton, além do
Django, sem alterar as versões já instaladas no sistema. Na pasta
\verb|exemplos/| se encontram alguns arquivos de configuração de
exemplo, pré-configurados para a conexão ao servidor LDAP da RM (ver
\ref{MocambosLDAP}).


\subsection{Django mmedia}
\framebox[\textwidth]{\footnotesize O código esta disponível no
  \url{https://github.com/RedeMocambos/mmedia}}

A aplicação \emph{mmedia} implementa o modelo de dados multimídia com
suporte par o armazenamento em repositório \emph{git-annex}.

Os modelos, em \emph{Django}, precisam ser criados no arquivo \verb|model.py|,
onde, primeiramente, definimos a estrutura de base da classe
\emph{MMedia}, com os atributos comuns para todos os objetos
multimídia. As classes \emph{Audio}, \emph{Image} e \emph{Video},
herdam os atributos da classe abstrata ``MMedia'', acrescentando outros
atributos específicos para o tipo de objeto.

Os objetos são salvos no \emph{database}, e serializados no disco
através o \emph{overriding} da função \emph{save()}:

\begin{code}
    def save(self, *args, **kwargs):
        logger.debug(type(self))
        serializeTo = os.path.join(settings.MEDIA_ROOT,\
                                   settings.GITANNEX_DIR,\
                                   settings.PORTAL_NAME,\
                                   settings.SERIALIZED_DIR,\
                                   os.path.basename(self.fileref.path)+ '.xml')
        logger.info('>>>> Serialize to: ' + serializeTo)
        out = open(serializeTo, "w")
        XMLSerializer = serializers.get_serializer("xml")
        xml_serializer = XMLSerializer()
        xml_serializer.serialize((self, ), stream=out)
        super(MMedia, self).save(*args, **kwargs)
\end{code}


\subsection{Django gitannex}
\framebox[\textwidth]{\footnotesize O código esta disponível no
  \url{https://github.com/RedeMocambos/gitannex}}

A aplicação \emph{gitannex} implementa parte do modelo de dados de um
repositório \emph{git-annex} no \emph{Django}, acrescentando os
atributos e as funcionalidades necessárias para a planejamento da
sincronização.

\begin{figure}[htbp]
  \centering
  \includegraphics[width=\textwidth]{./Figure/SequenceDiagram_NuovoOggetto-crop.pdf}
  \rule{35em}{0.5pt}
  \caption[Diagrama de sequencia da criação de um novo objeto
  multimídia]{Diagrama de sequencia da criação de um novo objeto
    multimídia.}
  \label{fig:SequenceDiagramAdd}
\end{figure}

Seguindo a especificação, é definido, através do atributo
\emph{syncStartTime}, um horário para o início da sincronização, que é
inicializada pela função \emph{runScheduledJobs()}. Para interagir
mais facilmente com o framework através do terminal é possível definir
comandos que precisam ser criados na pasta
\verb|management/commands/|, onde se encontra, por exemplo, o comando
\verb|run_scheduled_jobs| que chama a função homônima. Desta maneira
é possível executar operações planejando-as através do
\emph{cron}\footnote{``Em sistemas operacionais Unix e Unix-like, o
  comando crontab permite o planejamento de comandos, ou seja permite
  de agendá-los no sistema para serem executados periodicamente.'',
  traduzido de Wikipedia:
  \url{http://it.wikipedia.org/wiki/Crontab}.} diretamente no
terminal com:
\begin{verbatim}
zumbi@palmares:~$ python manage.py run_scheduled_jobs
\end{verbatim}

\emph{Django} providência um sistema de sinais, enviados em
concomitância de operações como a \emph{save()} de um objeto, que
podem ser interceptados em outras partes do sistema. A aplicação
\emph{gitannex} intercepta o sinal padrão do Django, \emph{post-save}
(ver figura \ref{fig:SequenceDiagramAdd}), em objetos que herdam da
classe \emph{MMedia}\footnote{Django não suporta a ``filtragem'' de
  sinais enviados por subclasses de uma classe dada, neste caso
  \emph{MMedia}. Para isso, foi usado um truque encontrado na rede
  (ver o código no arquivo \texttt{gitannex/signals.py}).}, através da
função \emph{gitMMediaPostSave()}:

\begin{code}
@receiver_subclasses(post_save, MMedia, ``mmedia_post_save'')
def gitMMediaPostSave(instance, **kwargs):
    logger.debug(instance.mediatype)
    logger.debug(type(instance))
    logger.debug(instance.path_relative())

    path = instance.path_relative().split(os.sep)
    if gitannex_dir in path:
        repositoryName = path[path.index(gitannex_dir) + 1]
        gitAnnexRep = GitAnnexRepository.objects.get(\
                      repositoryName__iexact=repositoryName)
        gitAnnexAdd(os.path.basename(instance.fileref.name),\
                    os.path.dirname(instance.fileref.path))
        gitCommit(instance.title, instance.author.username,\
                  instance.author.email, os.path.dirname(instance.fileref.path))
\end{code}
 % Prototipo

%\input{./Chapters/Chapter5} % 

%\input{./Chapters/Chapter6} % Results and Discussion

%\input{./Chapters/Chapter7} % Conclusion

%% ----------------------------------------------------------------
% Now begin the Appendices, including them as separate files

\addtocontents{toc}{\vspace{2em}} % Add a gap in the Contents, for aesthetics

\appendix % Cue to tell LaTeX that the following 'chapters' are Appendices

%% Appendice A                                                                                                                                                                   
\chapter{Listato del codice}
\label{AppendiceA}
\lhead{Appendice A \emph{Listato del codice}}

% \lstset{frame=tb,
%   language=Python,
%   aboveskip=3mm,
%   belowskip=3mm,
%   showstringspaces=false,
%   columns=flexible,
%   numbers=none,
%   numberstyle=\tiny\color{gray},
%   keywordstyle=\color{blue},
%   commentstyle=\color{dkgreen},
%   stringstyle=\color{mauve},
%   breaklines=true,
%   breakatwhitespace=true
%   tabsize=2
%   basicstyle={\scriptsize\ttfamily},
%   morekeywords={models, lambda, forms, =}
%   multicols=2
% }

\section{gitannex}

\subsection{gitannex/admin.py}
\lstinputlisting[basicstyle={\scriptsize\ttfamily}]{src/gitannex/admin.py}

\subsection{gitannex/models.py}
\lstinputlisting[basicstyle={\scriptsize\ttfamily}]{src/gitannex/models.py}

\subsection{gitannex/signals.py}
\lstinputlisting[basicstyle={\scriptsize\ttfamily}]{src/gitannex/signals.py}

\subsection{gitannex/management/commands/run scheduled jobs.py}
\lstinputlisting[basicstyle={\scriptsize\ttfamily}]{src/gitannex/management/commands/run_scheduled_jobs.py}


\section{mmedia}

\subsection{mmedia/admin.py}
\lstinputlisting[basicstyle={\scriptsize\ttfamily}]{src/mmedia/admin.py}

\subsection{mmedia/models.py}
\lstinputlisting[basicstyle={\scriptsize\ttfamily}]{src/mmedia/models.py}

\subsection{mmedia/signals.py}
\lstinputlisting[basicstyle={\scriptsize\ttfamily}]{src/mmedia/signals.py}

\subsection{mmedia/forms.py}
\lstinputlisting[basicstyle={\scriptsize\ttfamily}]{src/mmedia/forms.py}

\subsection{mmedia/management/commands/create objects from files.py}
\lstinputlisting[basicstyle={\scriptsize\ttfamily}]{src/mmedia/management/commands/create_objects_from_files.py}
	% Appendix Title

%\input{./Appendices/AppendixB} % Appendix Title

%\input{./Appendices/AppendixC} % Appendix Title

\addtocontents{toc}{\vspace{2em}}  % Add a gap in the Contents, for aesthetics
\backmatter

%% ----------------------------------------------------------------
\label{Bibliography}
\lhead{\emph{Bibliography}}  % Change the left side page header to "Bibliography"
\bibliographystyle{unsrtnat}  % Use the "unsrtnat" BibTeX style for formatting the Bibliography
\bibliography{Bibliography}  % The references (bibliography) information are stored in the file named "Bibliography.bib"

\end{document}  % The End
%% ----------------------------------------------------------------
