% Capitolo 4

\chapter{Conclusioni e sviluppi futuri}
\label{Capitolo5}
\lhead{Capitolo 5. \emph{Conclusioni e sviluppi futuri}}

L'architettura e il prototipo sviluppato saranno implementati su scala
ridotta, a partire dalle comunità più strutturate e che fungono già da
poli regionali, chiamati \emph{Nucleos de Formação Continuada, (NFC)},
nuclei di formazione continua, che sono attualmente dieci,
geograficamente ben distribuiti sul territorio nazionale brasiliano.

Seguendo la filosofia di sviluppo Agile \citep{Agile}, per cui: 

\begin{quote}
  ``\emph{Agile software development is a group of software
    development methods based on iterative and incremental
    development, where requirements and solutions evolve through
    collaboration between self-organizing, cross-functional
    teams.}''\footnote{``Lo sviluppo di software Agile è un insieme di
    metodi basati sullo sviluppo iterativo e incrementale, dove i
    requisiti e le soluzioni evolvono attraverso la collaborazione tra
    team auto-organizzati e multifunzionali.'', tratto da Wikipedia:
    \url{http://en.wikipedia.org/wiki/Agile_software_development}.}
\end{quote}

è importante, in un contesto cosi ampio e eterogeneo, avere una
struttura generale su cui poter sviluppare codice funzionante, da
poter evolvere e adattare nel tempo.

Il codice prodotto consente una prima implementazione per un progetto
pilota che coinvolga sviluppatori del NPDD e consenta la formazione di
tecnici locali nei NFC, oltre ad ottenere un feedback diretto sull'uso
da parte delle comunità.

Attualmente il codice supporta:
\begin{itemize}
\item autenticazione LDAP (con gestione basica dei gruppi)
\item creazione e upload di contenuti audio, immagini e video
\item distribuzione tramite \emph{git-annex}
\item sincronizzazione degli oggetti sui portali django (ricreando gli
  oggetti relativi ai contenuti distribuiti via \emph{git-annex})
\end{itemize}

Dal punto di vista strutturale la componente più importante è il
sistema di gestione delle identità digitali federate. Pertanto è
essenziale una sperimentazione approfondita della soluzione LDAP
adottata che può comunque essere estesa e integrata ad altri
meccanismi. Per la messaggistica e il VOIP, sarebbe interessante
esplorare le possibilità date dai protocolli XMPP, che inoltre si
presta per la facilità con cui può integrarsi in sistemi
eterogenei. Ad esempio l'integrazione tra XMPP e OpenID, per cui, una
volta autenticati, si ricevono le richieste di autorizzazione ad
accedere a siti terzi, tramite messaggi
istantanei\footnote{Un'implementazione pubblica di questo sistema è
  disponibile sul sito: \url{http://openid.xmpp.za.net/}.}. In questo
senso il trio LDAP, XMPP e OpenID coprirebbe le necessità di servizi
federati rispettivamente ``desktop'', ``client'' e ``web oriented''.

Nel frattempo il prototipo realizzato costituisce un primo passo, e la
prova su campo è un passaggio fondamentale per testare le reali
capacità e i limiti che sussistono. La distribuzione di file di grandi
dimensioni, oltre ai limiti di banda, comporta un notevole uso di
spazio su disco. È necessario quindi imporre dei limiti, ad esempio,
sulle dimensioni massime dei file accettate dal sistema.

Da implementare o migliorare nelle prossime versioni:
\begin{itemize}
\item trasferimento selettivo dei contenuti/valori basato sull'uso
  statistico o su richiesta
\item sviluppo di in interfaccia di visualizzazione e pubblicazione
  per l'utente finale
\item gestione del DIT dell'LDAP tramite script e/o portale
\end{itemize}

% \begin{figure}[htbp]
%   \centering
%   \includegraphics[width=\textwidth]{./Figure/SchemaServer_ReteMocambos-crop.pdf}
%   \rule{35em}{0.5pt}
%   \caption[Schema dell'infrastruttura della RM]{Schema dell'infrastruttura della RM.}
%   \label{fig:SchemaServer_ReteMocambos}
% \end{figure}
