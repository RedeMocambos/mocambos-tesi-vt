% Capitolo 5

\chapter{Conclusões e desenvolvimentos futuros}
\label{Capitulo5}
\lhead{Cap\'itulo 5. \emph{Conclusões e desenvolvimentos futuros}}

A arquitetura e o protótipo desenvolvidos são em curso de
implementação, em escada reduzida, a partir das comunidades mais
estruturadas que tem já um papel de polos regionais, chamados
\emph{Núcleos de Formação Continuada, (NFC)}, que são atualmente dez,
geograficamente bem espalhados no território brasileiro.

Seguindo a filosofia de desenvolvimento Agile \citep{Agile}, pela qual: 

\begin{quote}
  ``\emph{Agile software development is a group of software
    development methods based on iterative and incremental
    development, where requirements and solutions evolve through
    collaboration between self-organizing, cross-functional
    teams.}''\footnote{``O desenvolvimento de software Agile é um
    conjunto de metodologias baseadas no desenvolvimento iterativo e
    incremental, onde os requisitos e as soluções evolvem através a
    colaboração entre equipe auto-organizadas e multifuncionais.'',
    traduzido de Wikipedia:
    \url{http://en.wikipedia.org/wiki/Agile_software_development}.}
\end{quote}

é importante, em um contexto assim amplo e heterogêneo, ter uma
estrutura geral em cima da qual pode-se desenvolver código funcionante,
que possa ser melhorado e evoluído no tempo.

Atualmente o código suporta:
\begin{itemize}
\item autenticação LDAP (com gestão básica dos grupos)
\item criação e upload de conteúdos áudios, imagens e vídeos
\item distribuição através do \emph{git-annex}
\item sincronização dos objetos em portais Django (recriando os
  objetos relativos aos conteúdos distribuídos via \emph{git-annex})
\end{itemize}

Do ponto de vista estrutural a componente mais importante é o sistema
de gestão das identidades federadas digitais. Portanto é essencial uma
experimentação aprofundada da solução LDAP adotada que, contudo, pode
ser estendida e integrada com outros mecanismos. Para mensagens
instantâneas e VOIP, seria interessante explorar as possibilidades
oferecidas pelo protocolo XMPP, que se destaca pela facilidade de
integração em sistemas heterogêneos. Por exemplo a integração entre
XMPP e OpenID, onde, uma vez autenticados, se recebem os pedidos de
autorização a acessar a serviços terceiros, através de mensagens
instantâneas\footnote{Uma implementação pública deste sistema é
  disponível no site: \url{http://openid.xmpp.za.net/}.}. Neste
sentido o trio LDAP, XMPP e OpenID poderia oferecer uma boa estrutura
para serviços federados respetivamente ``desktop'', ``client'' e ``web
oriented''.

No que diz respeito ao protótipo, a parte implementada constitui um
primeiro passo, e a prova no campo é uma passagem fundamental para
testar as capacidades reais e os limites que ainda existem. A
distribuição de arquivos de grande tamanho, além dos limites de banda,
implica um considerável uso de espaço no disco. É necessário então
impor limites, por exemplo, nas dimensões máximas dos arquivos aceitos
pelo sistema. 

A ser implementados ou melhorado nas próximas versões:

\begin{itemize}
\item transferência seletiva dos conteúdos/valores baseado no uso
  estatístico o sob pedido
\item desenvolvimento de uma interface de visualização e publicação
  para o usuário final
\item gestão do DIT do LDAP através de \emph{script} e/ou portal
\end{itemize}

