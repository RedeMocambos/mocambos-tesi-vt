% Capitolo 3

\chapter{Un'architettura per la Rete Mocambos} % Write in your own chapter title
\label{Capitolo3}
\lhead{Capitolo 3. \emph{Un'architettura per la Rete Mocambos}} % Write in your own chapter title to set the page header

\section{Specifica dei requisiti}
La Rete Mocambos attualmente è composta da piu di 200 comunità
localizzate in tutto il territorio nazionale brasiliano e alcune
comunità in Africa e in Europa. La connettività in Brasile è
satellitare, garantita dal programma GESAC che mette a disposizione di ogni
comunità un Very Small Aperture Terminal (VSAT) con banda di 512
kbit/s in download e 128 kbit/s in upload. La topologia della rete è a
stella per cui tutti i nodi comunicano tramite satellite concentrando
il traffico su un hub terrestre, dove la rete satellitare è
interconnessa a Internet. Ogni comunità ha a disposizione una sala con
10 computer ad accesso pubblico (chiamati comunemente
``telecentros''), installati recentemente tramite il Telecentros.BR
\footnote{``Il programma nazionale di appoggio all'inclusione digitale
  nelle comunità, Telecentros.BR, è un'azione del Governo Federale di
  appoggio per la creazione di nuovi spazi pubblici e comunitari di
  inclusione digitale e per il potenziamento degli spazi già in
  funzione su tutto il territorio. Vengono messe a disposizione le
  apparecchiature informatiche e il mobiliario necessario al
  funzionamento dei \textit{telecentros}, servizi di connessione a
  internet tramite banda larga oltre a formazione e borse per
  l'appoggio economico per tutor che lavorano come agenti di
  inclusione digitale. Questi tutor borsisti partecipano ad un corso
  di formazione e lavorano, nei telecentros, per le comunità.'',
  tradotto da \url{http://www.inclusaodigital.gov.br/telecentros}.}, altro
programma federale brasiliano. La popolazione di ogni comunità varia
dalle centinaia alle migliaia di persone. La maggior parte delle
comunità si trova in area rurale, e spesso di difficile accesso. A
parte gli spazi comunitari, normalmente concentrati in una zona
centrale, la popolazione è divisa in piccoli nuclei familiari sparsi
sul territorio e a volte molto distanti tra di loro. 

Vediamo alcuni requisiti essenziali per questa architettura di rete
federata:

\subsection{Identità di rete}
Gli utenti delle comunità devono avere associate delle identità
digitali con cui accedere e utilizzare i servizi esistenti. Oltre ad
accedere localmente, è importante avere accesso ad alcuni servizi
anche al di fuori della propria comunità. Ad esempio, nel caso una
persona si trovi in città, deve poter accedere, attraverso internet,
ai servizi base, quale l'email, o ai portali e servizi web della
RM. È necessaria quindi un identità digitale di rete univoca.

\subsection{Autenticazione decentrata}
\label{sec:AutDec}
Un requisito essenziale per ogni comunità è l'autonomia in assenza di
collegamento ad internet, e la presenza quindi di un sistema locale di
autenticazione e autorizzazione. Oltre a garantire la continuità del
servizio (in relazione a problemi di connessione satellitare), è un
requisito importante per un buon disimpegno di tutti i servizi
autenticati.

\subsection{Sincronizzazione}
Lo scambio di informazioni tra le comunità è l'obbiettivo primario per
la RM. È quindi necessario un meccanismo per la sincronizzazione
selettiva dei contenuti di interesse generale. Inoltre, per
ottimizzare l'uso della banda, le operazioni di sincronizzazione dei
dati devono incidere il meno possibile sull'uso quotidiano di internet
e prevedere un sistema di agendamento notturno, o comunque quando la
connessione satellitare non è utilizzata. Il sistema deve prevedere
anche la possibilità di sincronizzare i dati da e verso periferiche di
archiviazione di massa.

\subsection{Riproducibilità}
Ogni comunità provvede autonomamente alla gestione della propria infrastruttura
tecnologica e le soluzioni adottate devono quindi essere facilmente
implementabili e adattabili localmente. Gli strumenti devono quindi essere
soluzioni stabili e possibilmente di uso comune, anche al fine di
reperire piu facilmente assistenza in loco. 

\subsection{Manutenzione}
La manutenzione del sistema deve prevedere sia interventi a distanza
sia locali. Il sistema deve essere basato su tecnologie standard e
aperte per garantire l'accesso ai dati anche in caso di problemi.

\subsection{Sviluppo}
Ogni comunità ha caratteristiche e necessità proprie che portano alla
richiesta di servizi differenziati. L'architettura generale della RM
deve essere una base stabile su cui poter sviluppare servizi senza
vincoli troppo stringenti. Le tecnologie e i linguaggi adottati devono
facilitare l'interazione, la formazione e il riuso delle conoscenze. 

\section{Strumenti e pratiche per lo sviluppo}
Data la natura della RM, e in particolare la necessita di autonomia
tecnologica, il lavoro svolto è orientato alla formazione per cui è
importante l'uso di strumenti che facilitino la documentazione e lo
sviluppo collettivo. La RM utilizza già da anni un wiki
\footnote{``Una Wiki è una pagina (o comunque una collezione di
  documenti ipertestuali) che viene aggiornata dai suoi utilizzatori e
  i cui contenuti sono sviluppati in collaborazione da tutti coloro
  che vi hanno accesso. La modifica dei contenuti è aperta, nel senso
  che il testo può essere modificato da tutti gli utenti (a volte
  soltanto se registrati, altre volte anche anonimi) contribuendo non
  solo per aggiunte come accade solitamente nei forum, ma anche
  cambiando e cancellando ciò che hanno scritto gli autori precedenti.
  Ogni modifica è registrata in una cronologia che permette in caso di
  necessità di riportare il testo alla versione precedente; lo scopo è
  quello di condividere, scambiare, immagazzinare e ottimizzare la
  conoscenza in modo collaborativo. Il termine wiki indica anche il
  software collaborativo utilizzato per creare il sito web e il
  server.'', tratto da \url{http://it.wikipedia.org/wiki/Wiki}.},
disponibile all'indirizzo \url{http://wiki.mocambos.net}.  Per il
versionamento del codice del lavoro svolto si è scelto di usare il
sistema decentrato GIT \ref{sec:GIT} e la piattaforma GITHUB
\footnote{\href{http://github.com}{GITHUB} è una rete sociale basata
  sul sistema GIT \ref{sec:GIT}.}. Il codice del prototipo è disponibile
all'indirizzo \url{https://github.com/RedeMocambos}.

\subsection{Sistema operativo}
La scelta di un sistema operativo comune facilita la documentazione,
l'automazione e l'assistenza a distanza. Molte comunità utilizzano già
la distribuzione GNU\\Linux Ubuntu \footnote{``Ubuntu è un sistema
  operativo GNU\/Linux nato nel 2004, basato su Debian, che si
  focalizza sull'utente e sulla facilità di utilizzo. Ubuntu è
  orientato all'utilizzo desktop e pone una grande attenzione al
  supporto hardware. È prevista una nuova versione ogni sei mesi.
  Finanziato dalla società Canonical Ltd (registrata nell'Isola di
  Man), questo sistema è rilasciato come software libero sotto licenza
  GNU GPL ed è gratuito e liberamente modificabile.'', tratto da
  \url{http://it.wikipedia.org/wiki/Ubuntu}.} per cui per il prototipo
è stata scelta l'ultima versione stabile 11.10.


\section{Architettura di base}

\subsection{Gestione delle identità di rete}
Analizzando la specifica dei requisiti e alcune delle tecnologie
esistenti viste nel capitolo precedente, è stato scelto l'uso di LDAP
per gestire le credenziali degli utenti all'interno della RM. Il
prototipo proposto prevede un server centrale su connettività internet
garantita e server locali in replica. È possibile pensare una
configurazione in cui tutti i server della rete sono configurati in
modalità N-Way-Multimaster. Questa configurazione, se da un lato
consente l'aggiornamento in scrittura della base utenti su ognuno dei
server, può generare più facilmente situazioni di inconsistenza dei
dati e problemi di sincronizzazione tra i server LDAP. Per questo
motivo per il prototipo realizzato si è scelto di rilassare il
requisito \ref{sec:AutDec}, ipotizzando un singolo server master per
tutta la rete su cui effettuare le operazioni in scrittura (creazione,
eliminazione e aggiornamento di utenti) e server locali abilitati solo
ad operazioni in lettura. Ogni comunità quindi avrebbe a disposizione
un server LDAP in replica su cui basare le operazioni di
autenticazione e autorizzazione. In mancanza di connessione esterna i
servizi locali rimangono comunque attivi per le utenze attive fino
all'ultima sincronizzazione.

Per facilitare l'implementazione da parte delle comunità dei server
locali LDAP è stato sviluppato uno script di installazione disponibile
su \url{https://github.com/RedeMocambos/Mocambos_LDAP}.

Lo \textit{script} bash provvede ad installare i pacchetti necessari e
preparare il server LDAP slapd in configurazione master o
replica. Inoltre crea un DIT preconfigurato per la RM con una semplice
struttura (vedi figura \ref{fig:DIT}).

\begin{verbatim}
# Create top-level object in domain
dn: dc=mocambos,dc=net
objectClass: top
objectClass: dcObject
objectclass: organization
o: Mocambos
dc: mocambos
description: LDAP Mocambos

# Admin user.
dn: cn=admin,dc=mocambos,dc=net
objectClass: simpleSecurityObject
objectClass: organizationalRole
cn: admin
description: LDAP administrator
userPassword: {SSHA}lx56Oal2lADo7y21hmy5GCdNWF7545Eh

dn: ou=people,dc=mocambos,dc=net
objectClass: organizationalUnit
ou: people

dn: ou=groups,dc=mocambos,dc=net
objectClass: organizationalUnit
ou: groups

dn: cn=coordenadores,ou=groups,dc=mocambos,dc=net
objectClass: groupOfNames
cn: coordenadores
member: uid=zumbi,ou=people,dc=mocambos,dc=net

dn: uid=zumbi,ou=people,dc=mocambos,dc=net
objectClass: inetOrgPerson
objectClass: posixAccount
objectClass: shadowAccount
uid: mocambola
sn: dos Palmares
givenName: Zumbi
cn: Zumbi dos Palmares
displayName: Zumbi dos Palmares
uidNumber: 5001
gidNumber: 50000
userPassword: {SSHA}lx56Oal2lADo7y21hmy5GCdNWF7545Eh
gecos: Zumbi dos Palmares
loginShell: /bin/bash
homeDirectory: /home/zumbi
shadowExpire: -1
shadowFlag: 0
shadowWarning: 7
shadowMin: 8
shadowMax: 999999
shadowLastChange: 10877
mail: zumbi@mocambos.net
ou: Quilombo dos Palmares
o: Mocambos
title: Liderança
initials: ZP

\end{verbatim}

