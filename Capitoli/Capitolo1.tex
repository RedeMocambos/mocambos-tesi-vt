% Capitolo 1

\chapter{Introduzione}
\label{Capitolo1}
\lhead{Capitolo 1. \emph{Introduzione}}

Questo lavoro parte da un esperienza diretta che ha avuto i natali nel
2004 a Firenze per poi svilupparsi e continuare in Brasile dal 2005 ad
oggi. Nel maggio del 2004, con il Collettivo di Informatica "Ada
Byron"\footnote{Il Collettivo di Informatica "Ada Byron", è un gruppo
  di studenti del Corso di Laurea in Informatica dell’Università degli
  Studi di Firenze. Il sito internet del collettivo è
  \url{http://informada.wordpress.com}.}, organizzammo una serie di
incontri e conferenze sul tema della libertà della conoscenza
applicata a vari ambiti. Parteciparono a "Modelli Liberi"\footnote{Un
  articolo sull'evento è disponibile su
  \url{http://www.apogeonline.com/webzine/2004/04/16/01/200404160103}.},
cosi si chiamarono gli incontri, relatori di rilievo internazionale,
quali Richard Stallman, Diego Saraiva, Juan Carlos Gentile e,
rappresentando il governo brasiliano, Sergio Amadeu e Elaine da
Silva. In seguito, nel 2005, mi trasferì a Brasilia dove iniziai a
lavorare al programma del governo federale brasiliano per
l'universalizzazione dell'accesso a internet, il GESAC (\emph{Governo
  Eletrônico Serviço de Atendimento ao Cidadão}). Il GESAC era, e
tuttora è, il più ambizioso progetto di lotta al \emph{digital divide}
dell'America Latina. Il programma prevedeva un numero iniziale di 3200
connessioni satellitari bidirezionali in tutto il territorio
nazionale, una piattaforma di servizi online e un equipe sul
territorio. Il Ministero delle Comunicazioni aveva affidato la
direzione del programma a Antonio Albuquerque, tra i fondatori del
Sindacato delle Telecomunicazioni brasiliano, che sostenne e diffuse
l'uso di Software Libero a tutti i livelli, sottolineandone
l'importanza strategica. Il nostro lavoro consisteva nella ricerca e
sperimentazione di soluzioni informatiche applicate ai più svariati
habitat e contesti sociali e culturali. Ci trovavamo continuamente a
contatto con realtà estremamente diverse. Il GESAC serve infatti
scuole, associazioni, caserme, comunità rurali, riserve indigene,
quilombo (Rif.), comunità di pescatori, ecc. Nella quasi totalità dei
casi erano persone alla prima esperienza informatica. L'approccio
scelto prediligeva il dialogo e il confronto aperto dove, ad esempio,
al posto di una lezione frontale, si ricorreva, ad esempio, ad una
\emph{roda de conversas}, tutti seduti in cerchio, pratica tra l'altro
molto popolare in Brasile. All'equipe del GESAC furono invitati a
lavorare persone con esperienza nell'ambito sociale, tra cui ad
esempio attivisti di Indymedia\footnote{``Indymedia e' un network di
  media gestiti collettivamente per una narrazione radicale, obiettiva
  e appassionata della verità. Ci impegniamo con amore e ispirazione
  per tutte quelle persone che lavorano per un mondo migliore, a
  dispetto delle distorsioni dei media che con riluttanza si impegnano
  a raccontare gli sforzi dell'umanità libera.'', tratto da
  http://www.indymedia.org/it/static/about.shtml .} e
Intervozes\footnote{``Intervozes è un'organizzazione che lavora per
  rendere effettivo il diritto umano alla comunicazione in Brasile.
  Per Intervozes, il diritto alla comunicazione è indissociabile dal
  pieno esercizio di cittadinanza e democrazia. Una società può essere
  definita democratica solo quando le diverse voci, opinioni e culture
  che la compongono hanno spazio per manifestarsi.'', tradotto da
  http://www.intervozes.org.br/o-intervozes .}. Il dialogo aperto e
paritario, a partire dalle specificità del contesto, apriva nuovi
spazi e punti di vista per la rivoluzione digitale, rimettendo al
centro della discussione il fine oltre che il mezzo. Non solo non
portavamo soluzioni preconfezionate, ma spesso gli strumenti e le
pratiche dovevano essere ridiscusse e possibilmente adattate a nuove
necessità.

\section{Neutralità tecnologica}
Questa tesi ricerca una soluzione tecnologica per facilitare la
comunicazione tra comunità quilombola. Queste comunità
afro-discendenti condividono tra loro molti aspetti sociali,
culturali, storici e geografici che le differenziano dalle realtà per
la quale, e dalla quale, sono sviluppati normalmente i mezzi di
comunicazione digitale. La neutralità del mezzo viene troppo spesso
data per scontata e risulta difficile percepire quanto questo in
realtà influisca e condizioni le nostre azioni e i nostri
obiettivi. Siamo portati a pensare al mezzo come neutrale, ma se
prendiamo ad esempio i principali e più diffusi mezzi di
comunicazione, le lingue, possiamo intuire come queste non siano
interscambiabili essendo l'espressione delle culture e delle società
che le usano e le vivono. Nell'era digitale è importante considerare
il peso e l'influenza dei nuovi linguaggi e mezzi di comunicazione.

\section{Internet, autonomia e libertà tecnologica}
Negli ultimi anni la diffusione della banda larga, ma sopratutto
strategie come quella adottata da Google, hanno trasformato il
concetto stesso di internet che da rete globale di reti eterogenee,
diventa principalmente una rete per la globalizzazione di servizi
fortemente centralizzati e uniformati. Fino a pochi anni fa era
normale per un'organizzazione provvedere alla gestione, mantenimento e
a volte allo sviluppo di servizi oltre che della infrastruttura
tecnologica per i propri utenti, a partire dalle basi di dati fino ai
servizi per la messaggistica. In questi contesti sono nati e si sono
sviluppate buona parte delle tecnologie di comunicazione oggi
disponibili ma che vengono in parte accantonate per scelte sopratutto
di mercato. Internet è nata dal confronto aperto tra i responsabili di
diverse reti, uniti dalla volontà di connettere le loro differenti
realtà. La nascita di nuovi servizi, per queste reti eterogenee, era
basata sulla discussione e il confronto. Il primo uso documentato del
termine "internet", seguendo questa pratica, fa la sua comparsa
proprio in un RFC \citep{RFC675}. I Request for comments (RFC), sono
dei documenti, normalmente scritti in un linguaggio semplice e
informale, mirati alla diffusione e discussione di nuove tecnologie
nell'ambiente delle telecomunicazioni. Sono anche la principale base
per la definizione di standard e protocolli. Attualmente è il canale
ufficiale di pubblicazione del \emph{The Internet Engineering Task
  Force (IETF)}, del \emph{The Internet Architecture Board (IAB)}, tra
le altre, e, in generale, della comunità mondiale dei ricercatori del
campo delle reti di comunicazione. Internet quindi, nata dalla pratica
della discussione, del confronto, della ``richiesta di commenti'', sta
ultimamente vivendo l'era dei \emph{Terms of Service (ToS)}. Ai
protocolli standard si sostituiscono API (Rif.) proprietarie e
suscettibili di alterazioni continue e unilaterali. Un impresa offre
servizi tramite API, decidendo i termini con cui gli utenti, o altre
imprese, possono accedere e utilizzare questi servizi.

In sostanza un tempo lo sviluppo seguiva un modello \emph{bottom up}, per
cui le maestranze informatiche sviluppavano sistemi ad hoc per le
esigenze locali, per poi in seguito aprire una discussione in rete,
con i loro corrispettivi, per definire dei protocolli standard e
mettere in comunicazione il tutto.

Oggi si passa ad un modello di sviluppo \emph{top down} per cui nuovi
servizi vengono lanciati basandosi su indagini di mercato e test su
campioni di utenti. Poche imprese fanno da padrone nel tracciare lo
sviluppo della rete che assume orizzonti sempre più determinati dal
mercato. Inoltre il mercato di riferimento, e il campione scelto, sono
per lo più legati al contesto economico, sociale e culturale
nordamericano e europeo. Pur essendo un analisi di interesse
filosofico/antropologico diventa territorio di frontiera con
l'informatica studiare la cultura digitale e gli effetti che le
tecnologie digitali possono avere sulle culture di paesi e popoli
diversi. Una analisi strutturata e analitica ci è stata lasciata da
Vilém Flusser, recentemente riscoperto e considerato tra i primi
filosofi della comunicazione dei nostri tempi, di cui riporto due
pensieri molto attuali, anche se pubblicati già nel 1983 nel libro
\emph{F{\"u}r eine Philosophie der Fotografie}
\citep{flusser1983philosophie}.


\begin{quote}
  ``\emph{Both those taking snaps and documentary photographers,
    however, have not understood 'information.' What they produce are
    camera memories, not information, and the better they do it, the
    more they prove the victory of the camera over the human being.}''
\end{quote}

e 

\begin{quote}
  ``\emph{Our thoughts, feelings, desires and actions are being
    robotized; 'life' is coming to mean feeding apparatuses and being
    fed by them. In short: Everything is becoming absurd. So where is
    there room for human freedom?}''
\end{quote}


Un approccio diametralmente opposto alla centralizzazione della rete
sono le tecnologie Peer To Peer (P2P). Il principio base del P2P
prevede una infrastruttura di comunicazione in cui i nodi comunicano
direttamente tra di loro senza nodi centrali preconfigurati. Esistono
anche protocolli P2P che prevedono dei super-nodi, che in parte
riprendono il modello client/server, normalmente usati per ottimizzare
le prestazioni e risolvere problemi di attraversamento di reti
mascherate. In generale i sistemi P2P sono auto-configuranti o
necessitano di una conoscenza minima dell'architettura della rete
sottostante, dato che si basano su protocolli per l'instradamento
automatico dei pacchetti. Queste tecnologie funzionano bene quando un
numero sufficiente di nodi è attivo e partecipa al funzionamento della
rete. Un esempio di protocollo P2P molto diffuso è bittorrent, con cui
è possibile trasferire dati ad alte velocità quando i contenuti
richiesti sono presenti su nodi con banda a disposizione.

Nonostante le tecnologie P2P siano un alternativa disponibile e già
implementata a vari livelli, presentano alcune caratteristiche per cui
non sono applicabili al contesto della Rete Mocambos (RM) (vedi
\ref{sec:ReteMocambos}). Le connessioni tra le comunità della RM sono
tutte via satellite, con banda molto limitata. Un protocollo P2P,
gravando sulla banda di molti nodi per operazioni operate da un
singolo nodo, penalizzerebbe proprio la risorsa più scarsa.

Le necessità di una rete come la Rete Mocambos sono di vario tipo e gestite da
entità differenti sebbene federate. Nel prossimo capitolo vedremo un
tipo di rete federata per la RM. 

