% Capitulo 1

\chapter{Introdução}
\label{Capitulo1}
\lhead{Cap\'itulo 1. \emph{Introdu\c{c}\~ao}}


Este trabalho é baseado numa experiencia direta que teve origem em
2004 em Florença para depois se desenvolver e continuar no Brasil
desde 2005 ate hoje. No maio de 2004, com o Coletivo de Informática
``Ada Byron''\footnote{O Coletivo de Informática ``Ada Byron'' é um
  grupo de estudantes do Curso de Informática da Universidade de
  Florença. O sitio internet do coletivo é
  \url{http://informada.wordpress.com}}, e com a associação
Hipatia\footnote{Hipatia nasceu como coordenamento espontaneo de
  pessoas do mundo inteiro que compartilham a mesma visão e o mesmo
  objetivo: uma sociedade do conhecimento global baseada na liberdade,
  igualdade e soliedaridade'', traduzido desde
  \url{http://www.hipatia.net}.}, organizamos uma serie de encontros e
conferencias sobre o tema da liberdade do conhecimento aplicada a
vários ámbitos. Parteciparam a ``Modelli liberi''\footnote{``Modelos
  livres''. Um artigo sobre o evento é disponível, em italiano, em
  \url{http://www.apogeonline.com/webzine/2004/04/16/01/200404160103}},
assim se chamavam os encontros, relatores de relevança internacional,
como Richard Stallman\footnote{Richard Stallman é presidente e
  fundador da \emph{Free Software Foundation}}, Diego
Saraiva\footnote{Professor da Universidade Nacional de Salta em
  Argentina e fundador do projeto Ututo, \url{http://ututo.org}}, Juan
Carlos Gentile\footnote{Juan Carlos Gentile é coordenador e um dos
  fundadores da ONG Hipatia, \url{http://www.hipatia.net}.} e,
representando o governo brasileiro, Sergio Amadeu\footnote{Sergio
  Amadeu é professor da Universidade Federal do ABC (UFABC), e na
  epoca era presidente do Instituto Nacional de Tecnologia da
  Informação do Brasil.} e Elaine da Silva\footnote{Elaine da Silva
  Tozzi é diretora de cultura do Municipio de Hortolandia/SP, na epoca
  era na coordenação do GESAC.}. Em seguida, retomei os contatos com
Elaine, e no 2005 me mudei para Brasília onde comecei a trabalhar no
programa de inclusão digital federal GESAC, \emph{Governo Eletrônico
  Serviço de Atendimento ao Cidadão}. O GESAC, era um dos mais
ambiciosos projetos de luta a exclusão digital da América Latina. O
programa previa um numero inicial de 3200\footnote{Hoje as
  conexões são mais de 11000.} conexões por satélite, uma plataforma
de serviços online e uma equipe no território. O Ministério das
Comunicações tinha entregue a direção do programa para Antonio
Albuquerque, um dos fundadores do Sindicado das Telecomunicações
brasileiro, que apoiou e espalhou o uso do Software Livre em todos os
níveis do programa, remarcando a sua importância estratégica. O nosso
trabalho consistia na pesquisa e experimentação de soluções
informáticas aplicadas aos mais variados habitat e contextos sociais e
culturais. Eramos continuamente em contato com realidades extremamente
diferentes. O GESAC atende escolas, associações, quarteis, comunidades
rurais, aldeias indígenas, quilombos\footnote{Os quilombos são
  comunidades fundadas por africanos deportados e afrodescendentes que
  resistiram a escravidão perpetrada por colonizadores portugueses e
  europeus em toda América Latina. Muitas das comunidades chegaram ate
  os dias de hoje e, só no Brasil, se contam cerca de cinco mil. A
  Constituição brasileira estabelece o direito a terra para os
  quilombolas, também no sentido de reparação histórica.}, caiçaras,
etc. Na maioria dos casos eram pessoas na primeira experiencia
informática. A abordagem escolhida favorecia o dialogo e a troca, como
por exemplo, no lugar de aulas, se recorria a rodas de conversa. Na
equipe do GESAC foram convidados a trabalhar pessoas com experiencia
na área social, como por exemplo ativistas de Indymedia\footnote{``O
  CMI Brasil é uma rede de produtores e produtoras independentes de
  mídia que busca oferecer ao público informação alternativa e crítica
  de qualidade que contribua para a construção de uma sociedade livre,
  igualitária e que respeite o meio ambiente.'', retirado de
  \url{http://www.midiaindependente.org/pt/blue/static/about.shtml}.},
e Intervozes\footnote{A missão do Intervozes é: ``promover o direito
  humano à comunicação, trabalhando para que este seja apropriado e
  exercido pelo conjunto da sociedade na luta por uma sociedade
  democrática, justa e libertária, construída por meio da autonomia,
  dignidade e participação de todos e todas., retirado de
  \url{http://www.intervozes.org.br/o-intervozes}.}. O dialogo aberto
e paritário, a partir das especificidades do contexto, abria novos
espaços e olhares para a revolução digital, recolocando ao centro da
discussão o fim, além do meio. Além de não chegar com soluções
pré-cozidas, frequentemente as ferramentas e as praticas precisavam
ser reinterpretadas e possivelmente adaptadas as novas necessidades.

A essa experiencia seguiu a colaboração com algumas das comunidades
que eram atendidas pelo GESAC, em especial com a Casa de Cultura
Tainã\footnote{A Casa de Cultura Tainã é uma entidade cultural e
  social sem fins lucrativos fundada por moradores da periferia de
  Campinas em 1989.  O objetivo da entidade é possibilitar o acesso à
  informação, fortalecendo a prática da cidadania e a formação da
  identidade cultural, visando contribuir para a formação de
  indivíduos conscientes e atuantes na comunidade. O endereço do sitio
  internet é: \url{http://www.taina.org.br}.} que na naqueles anos
colocava as bases do primeiro núcleo da Rede Mocambos (RM) (ver
\ref{RedeMocambos}).

\section{Neutralidade tecnológica}
Essa monografia pesquisa uma solução tecnológica para facilitar a
comunicação entre comunidades quilombolas. Essas comunidades
afrodescendentes compartilham muitos aspetos sociais, culturais,
históricos e geográficos que as diferenciam das realidades para quais,
e pelas quais, são normalmente desenvolvidos os meios de comunicação
digital. A neutralidade do meio é muitas vezes tida como certa, assim
fica difícil perceber quanto este na realidade influi e condicione
nossas ações e nossos objetivos. Somos levados a pensar o meio como
neutral, mas se consideramos, como exemplo, os principais e mais comuns
meios de comunicação, as línguas, podemos perceber como essas não são
intercambiáveis sendo a expressão das culturas e das sociedades que as
usam e as vivem. Na era digital é importante considerar o peso e a
influencia das novas linguagens e meios de comunicação. 

\section{Internet, autonomia e liberdade tecnológica}
Nos últimos anos a difusão da banda larga, mas sobretudo estratégias
como aquela adotada pela Google, transformaram o próprio conceito da
internet que de rede global de redes heterogêneas, vira principalmente
uma rede para a globalização de serviços fortemente centralizados e
uniformados. Ate poucos anos atras era normal para uma organização
prover a gestão, manutenção e as vezes ao desenvolvimento de serviços
alem da infraestrutura tecnológica para os próprios usuários, desde as
bases de dados ate aos serviços de alto nível, como sistemas para
mensagens e armazenamento de arquivos. Nesses contextos nasceram e se
desenvolveram boa parte das tecnologias de comunicação hoje
disponíveis mas que vem sendo em parte abandonadas por escolhas
principalmente de mercado. Internet nasceu da troca aberta entre os
responsáveis de varias redes, unidos pela vontade de conectar as suas
diferentes realidades. A criação de novos serviços, para essas redes
heterogêneas, era baseada na discussão e na confrontação. O primeiro
uso documentado do termo ``internet'', seguindo essa pratica, apareceu
de facto num RFC \citep{RFC675}. Os ``Request for Comments
(RFC)''\footnote{Pedido de comentários.} são documentos, normalmente
escritos em uma linguagem simples e informal, orientados para a
difusão e discussão de novas tecnologias no ambiente das
telecomunicações. São também a base principal para a definição de
novos padrões e protocolos. Atualmente os RFC são o canal oficial de
publicação da \emph{The Internet Engineering Task Force (IETF)}, da
\emph{The Internet Architecture Board (IAB)}, entre outros, e, em
geral, da comunidade mundial de pesquisadores da área das redes de
comunicação.

Internet então, rede de redes heterogêneas, nascida da pratica da
discussão, da confrontação, do ``pedido de comentários'', esta
ultimamente vivendo a era dos \emph{Terms of Service
  (ToS)}\footnote{Termo de Uso} sobre infraestruturas sempre mais
homologadas e centralizadas, as ditas ``nuvens'', ou, mais geralmente
e comercialmente, \emph{cloud}\footnote{``Em informática, com o termo
  inglês, cloud computing se indica um conjunto de tecnologias que
  permitem, normalmente sob forma de serviço oferecido por um provedor
ao cliente, de memorizar/armazenar e/ou elaborar dados (através de CPU
ou software) graças ao uso de recursos hardware/software disponíveis e
virtualizados na Rede. O uso correto do termo é contestado por vários
expertos: se essas tecnologias são vistas por alguns como uma maior
evolução tecnológica oferecida pela rede Internet, por outros, como
Richard Stallman, são considerados uma armadilha de marketing.'',
traduzido de Wikipedia:
\url{http://it.wikipedia.org/wiki/Cloud_computing}.}. Aos protocolos
padrão se substituem API proprietárias e suscetíveis de alterações
continuas e unilaterais. Uma empresa oferece serviços através API,
decidindo os termos pelos quais os usuários, ou outras empresas, podem
acessar e usar estes serviços.  

Do site do Google:
\begin{quote}
  ``\emph{Google has been pushing the technological bounds of cloud
    computing for more than ten years. Today, feedback and usage
    statistics from hundreds of millions of users in the real world
    help us bring stress-tested innovation to business customers at an
    unprecedented pace. From our consumer user base, we quickly learn
    which new features would be useful in the business context, refine
    those features, and make them available to Google Apps customers
    with minimal delay.}''\footnote{``Google tem empurrado os limites
    tecnológicos do cloud computing desde mais de dez anos. Hoje, os
    feedback e as estatísticas de uso de centenas de milhões de
    usuários reais no mundo nos ajudam a trazer inovações bem
    experimentadas para os nossos clientes comerciais com um ritmo sem
    precedentes.  Desde a base de usuários, aprendemos rapidamente
    quais características podem ser uteis para o mercado,
    melhorando-as e deixando-as disponíveis para os usuários de Google
    Apps com um atraso minimo.'', Retirado desde
    \url{http://www.google.com/apps/intl/en/business/cloud.html}.}
\end{quote}

Como declarado ainda, ``o Cloud computing esta no DNA do Google'' que
tem mais de 350 milhões de usuários ativos na sua
nuvem\footnote{``Declaração dada no ``earning call'' do 19 janeiro de
  2012, ver:
  \url{http://thenextweb.com/google/2012/01/19/gmail-closes-in-on-hotmail-with-350-mm-active-users/}.}.
 
Segundo a sociedade de consultoria Gartner, em 2016, todas as empresas
contempladas pelo \emph{Forbes Global 2000} usaram soluções
\emph{cloud} \citep{EY2011}.

Em essência um tempo o desenvolvimento seguia um modelo \emph{bottom
  up}\footnote{De baixo para cima.}, no qual os mestres informáticos
desenvolviam sistemas ad hoc para as necessidades locais, para depois
abrir uma discussão em rede com seus homólogos, para definir protocolos
padrão e colocar em comunicação o todo. 

Hoje passamos a um modelo de desenvolvimento \emph{top
  down}\footnote{De cima pra baixo.}, no qual novos serviços são
lançados se baseando em pesquisas de mercado e testes sobre amostras
de usuários. Pocas empresas dominam em traçar o desenvolvimento da
rede que assume horizontes sempre mais determinados pelo
mercado. Além disso o mercado de referencia, e amostra escolhida, são
principalmente ligados a contexto econômico, social e cultural
norte-americano e europeu.   

Mesmo sendo uma analise de interesse filosófico/antropológico vira
território de fronteira com a informática estudar a cultura digital e
os efeitos que as tecnologias digitais podem ter sobre as culturas de
diversos países e povos. Uma analise estruturada e analítica foi
deixada pelo Vilém Flusser, recentemente redescoberto e considerado
entre os primeiros filósofos da comunicação dos nossos tempos, de quem
cito dois pensamentos muito atuais, mesmo se publicados já no 1983 no
livro \emph{F{\"u}r eine Philosophie der Fotografie}:

\begin{quote}
  ``\emph{Both those taking snaps and documentary photographers, however,
  have not understood 'information.' What they produce are camera
  memories, not information, and the better they do it, the more they
  prove the victory of the camera over the human
  being.}''\footnote{``Seja quem se limita a tirar uma fotografia,
  seja os fotógrafos documentaristas, não compreenderam a
  'informação'. Aquilo que produzem são memorias fotográficas, não
  informação, e melhor o fazem, mais confirmam a vitoria da maquina
  fotográfica sobre o ser humano.''}, \citet{flusser1983philosophie}.
\end{quote}

e 

\begin{quote}
  ``\emph{Our thoughts, feelings, desires and actions are being
    robotized; 'life' is coming to mean feeding apparatuses and being
    fed by them. In short: everything is becoming absurd. So where is
    there room for human freedom?}''\footnote{``Nossos pensamentos,
    sentimentos, desejos e ações vem vendo robotizados; a 'vida'
    começa a significar nutrir aparelhos e ser nutridos por eles. Em
    breve: tudo esta tornando-se absurdo. Assim onde esta o espaço
    para a liberdade humana?''}, \citet{flusser1983philosophie}.
\end{quote}

Uma abordagem diametralmente oposta a centralização da rede são as
tecnologias \emph{Peer To Peer (P2P)}\footnote{O principio base do P2P
  prevê uma infraestrutura de comunicação onde os nós comunicam
  diretamente entre eles, sem nós centrais pré-configurados. Existem
  também protocolos P2P que prevêem super-nós, de alguma forma
  replicando o modelo cliente/servidor, normalmente usados para
  otimizar as prestações e superar problemas de atravessamento de
  redes mascaradas. Geralmente os sistemas P2P não precisam de
  configuração e necessitam um conhecimento minimo da arquitetura
  subjacente, dado que se baseiam em protocolos para roteamento
  automático dos pacotes. Essas tecnologias funcionam bem quando um
  numero suficiente de nós é ativo e participa ao funcionamento da
  rede. Um exemplo de protocolo P2P muito usado é \emph{bittorrent},
  através do qual é possível transferir dados em alta velocidade
  quando os conteúdos requeridos são presentes em nós com banda a
  disposição.}.

Apesar que as tecnologias P2P sejam uma alternativa disponível e já
implementada a vários níveis, apresentam algumas caraterísticas pelas
quais não podem ser amplamente aplicadas ao contexto da Rede
Mocambos. As conexões das comunidades da RM são quase todas via
satélite, com banda muito limitada. Os protocolos P2P, pesando na
banda de muitos nós, para operações efetuadas por um único nó,
penalizariam exatamente o recurso mais escasso.  


