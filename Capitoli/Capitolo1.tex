% Capitulo 1

\chapter{Introdução}
\label{Capitulo1}
\lhead{Capitulo 1. \emph{Introdução}}


Este trabalho começa a partir de uma experiencia direta que teve
origem em 2004 em Florença para depois se desenvolver e continuar no
Brasil desde 2005 ate hoje. No maio de 2004, com o Coletivo de
Informática ``Ada Byron''\footnote{O Coletivo de Informática ``Ada
  Byron'' é um grupo de estudantes do Curso de Informática da
  Universidade de Florença. O sitio internet do coletivo é
  \url{http://informada.wordpress.com}}, e com a associação
Hipatia\footnote{Hipatia nasceu como coordenamento espontaneo de
  pessoas do mundo inteiro que compartilham a mesma visão e o mesmo
  objetivo: uma sociedade do conhecimento global baseada na liberdade,
  igualdade e soliedaridade'', traduzido desde
  \url{http://www.hipatia.net}.}, organizamos uma serie de encontros e
conferencias sobre o tema da liberdade do conhecimento aplicada a
vários ámbitos. Parteciparam a ``Modelli liberi''\footnote{``Modelos
  livres''. Um artigo sobre o evento é disponível, em italiano, em
  \url{http://www.apogeonline.com/webzine/2004/04/16/01/200404160103}},
assim se chamavam os encontros, relatores de relevança internacional,
como Richard Stallman\footnote{Richard Stallman é presidente e
  fundador da \emph{Free Software Foundation}}, Diego
Saraiva\footnote{Professor da Universidade Nacional de Salta em
  Argentina e fundador do projeto Ututo, \url{http://ututo.org}}, Juan
Carlos Gentile\footnote{Juan Carlos Gentile é coordenador e um dos
  fundadores da ONG Hipatia, \url{http://www.hipatia.net}.} e,
representando o governo brasileiro, Sergio Amadeu\footnote{Sergio
  Amadeu é professor da Universidade Federal do ABC (UFABC), e na
  epoca era presidente do Instituto Nacional de Tecnologia da
  Informação do Brasil.} e Elaine da Silva\foonote{Elaine da Silva
  Tozzi é diretora de cultura do Municipio de Hortolandia/SP, na epoca
  era na coordenação do GESAC.}. Em seguida, retomei os contatos com
Elaine e em 2005 me mudei para Brasília onde começei a trabalhar no
programa de inclusão digital do governo federal brasileiro, o GESAC
(\emph{Governo Eletrônico Serviço de Atendimento ao Cidadão}). O
GESAC, era, e ainda é, um dos mais ambiciosos projetos de luta a
exclusão digital da América Latina. O programa previa um numero
inicial de 3200\footnote{Ate hoje as conexões são mais de 11000.}
conexões satelitares, uma plataforma de serviços online e uma equipe
no território. O Ministério das Comunicações tinha 


=================================================================================





Il programma prevedeva un
  numero iniziale di 3200\footnote{Ad oggi le connessioni sono più di
    11000.} connessioni satellitari bidirezionali in tutto il
  territorio nazionale, una piattaforma di servizi online e un'equipe
  sul territorio. Il Ministero delle Comunicazioni aveva affidato la
  direzione del programma a Antonio Albuquerque, tra i fondatori del
  Sindacato delle Telecomunicazioni brasiliano, che sostenne e diffuse
  l'uso di Software Libero a tutti i livelli, sottolineandone
  l'importanza strategica. Il nostro lavoro consisteva nella ricerca e
  sperimentazione di soluzioni informatiche applicate ai più svariati
  habitat e contesti sociali e culturali. Ci trovavamo continuamente a
  contatto con realtà estremamente diverse. Il GESAC serve infatti
  scuole, associazioni, caserme, comunità rurali, riserve indigene,
  quilombo\footnote{I Quilombo sono comunità fondate da africani
    deportati e afro-discendenti che resistettero e si rivoltarono
    alla schiavitù perpetrata dai colonizzatori portoghesi e europei
    in tutta l'America Latina. Molte delle comunità sono arrivate ai
    giorni nostri e, solo in Brasile, se ne contano circa
    cinquemila. La costituzione brasiliana sancisce il diritto alla
    terra per i Quilombo, anche in segno di riparazione storica. I
    Quilombo sono per la maggior parte localizzati in area rurale e
    spesso di difficile accesso.}, comunità di pescatori, ecc. Nella
  quasi totalità dei casi erano persone alla prima esperienza
  informatica. L'approccio scelto prediligeva il dialogo e il
  confronto aperto dove, ad esempio, al posto di una lezione frontale,
  si ricorreva ad una \emph{roda de conversas}, tutti seduti in
  cerchio, pratica tra l'altro molto popolare in Brasile. All'equipe
  del GESAC furono invitati a lavorare persone con esperienza
  nell'ambito sociale, tra cui ad esempio attivisti di
  Indymedia\footnote{``Indymedia e' un network di media gestiti
    collettivamente per una narrazione radicale, obiettiva e
    appassionata della verità. Ci impegniamo con amore e ispirazione
    per tutte quelle persone che lavorano per un mondo migliore, a
    dispetto delle distorsioni dei media che con riluttanza si
    impegnano a raccontare gli sforzi dell'umanità libera.'', tratto
    da \url{http://www.indymedia.org/it/static/about.shtml}.} e
  Intervozes\footnote{``Intervozes è un'organizzazione che lavora per
    rendere effettivo il diritto umano alla comunicazione in Brasile.
    Per Intervozes, il diritto alla comunicazione è indissociabile dal
    pieno esercizio di cittadinanza e democrazia. Una società può
    essere definita democratica solo quando le diverse voci, opinioni
    e culture che la compongono hanno spazio per manifestarsi.'',
    tradotto da \url{http://www.intervozes.org.br/o-intervozes}.}. Il
  dialogo aperto e paritario, a partire dalle specificità del
  contesto, apriva nuovi spazi e punti di vista per la rivoluzione
  digitale, rimettendo al centro della discussione il fine oltre che
  il mezzo. Non solo non portavamo soluzioni preconfezionate, ma
  spesso gli strumenti e le pratiche dovevano essere ridiscusse e
  possibilmente adattate a nuove necessità.

A questa esperienza, seguì la collaborazione con alcune delle comunità
che erano servite dal GESAC, in particolare con la \emph{Casa de
  Cultura Tainã}\footnote{La \emph{Casa de Cultura Tainã} è un'entità
  sociale e culturale senza fini di lucro fondata nel 1989 da abitanti
  della periferia di Campinas. Obbiettivo dell'entità è rendere
  possibile l'accesso all'informazione fortificando la pratica di
  cittadinanza e la formazione dell'identità culturale, al fine di
  contribuire nella crescita di individui coscienti e attuanti nella
  comunità. L'indirizzo del sito internet è
  \url{http://www.taina.org.br}.} che proprio in quegli anni poneva le
basi del primo nucleo della Rete Mocambos (RM) (vedi
\ref{sec:ReteMocambos}). 

\section{Neutralità tecnologica}
Questa tesi ricerca una soluzione tecnologica per facilitare la
comunicazione tra comunità quilombola. Tali comunità
afro-discendenti condividono tra loro molti aspetti sociali,
culturali, storici e geografici che le differenziano dalle realtà per
la quale, e dalla quale, sono sviluppati normalmente i mezzi di
comunicazione digitale. La neutralità del mezzo viene troppo spesso
data per scontata e risulta difficile percepire quanto questo in
realtà influisca e condizioni le nostre azioni e i nostri
obiettivi. Siamo portati a pensare al mezzo come neutrale, ma se
prendiamo ad esempio i principali e più diffusi mezzi di
comunicazione, le lingue, possiamo intuire come queste non siano
interscambiabili essendo l'espressione delle culture e delle società
che le usano e le vivono. Nell'era digitale è importante considerare
il peso e l'influenza dei nuovi linguaggi e mezzi di comunicazione.

\section{Internet, autonomia e libertà tecnologica}
Negli ultimi anni la diffusione della banda larga, ma sopratutto
strategie come quella adottata da Google, hanno trasformato il
concetto stesso di internet che da rete globale di reti eterogenee,
diventa principalmente una rete per la globalizzazione di servizi
fortemente centralizzati e uniformati. Fino a pochi anni fa era
normale per un'organizzazione provvedere alla gestione, mantenimento e
a volte allo sviluppo di servizi oltre che della infrastruttura
tecnologica per i propri utenti, dalle basi di dati fino ai servizi di
alto livello, quali messaggistica e accesso a \emph{file storage}. In
questi contesti sono nate e si sono sviluppate buona parte delle
tecnologie di comunicazione oggi disponibili ma che vengono in parte
accantonate per scelte sopratutto di mercato. Internet è nata dal
confronto aperto tra i responsabili di diverse reti, uniti dalla
volontà di connettere le loro differenti realtà. La nascita di nuovi
servizi, per queste reti eterogenee, era basata sulla discussione e il
confronto. Il primo uso documentato del termine "internet", seguendo
questa pratica, fa la sua comparsa proprio in un RFC \citep{RFC675}. I
Request for comments (RFC), sono dei documenti, normalmente scritti in
un linguaggio semplice e informale, mirati alla diffusione e
discussione di nuove tecnologie nell'ambiente delle
telecomunicazioni. Sono anche la principale base per la definizione di
standard e protocolli. Attualmente gli RFC sono il canale ufficiale di
pubblicazione del \emph{The Internet Engineering Task Force (IETF)},
del \emph{The Internet Architecture Board (IAB)}, tra le altre, e, in
generale, della comunità mondiale dei ricercatori del campo delle reti
di comunicazione.

Internet quindi, rete di reti eterogenee, nata dalla pratica della
discussione, del confronto, della ``richiesta di commenti'', sta
ultimamente vivendo l'era dei \emph{Terms of Service (ToS)} su
infrastrutture sempre più omologate e centralizzate, le cosiddette
``nuvole'', o, più comunemente e commercialmente,
\emph{cloud}\footnote{``In informatica, con il termine inglese cloud
  computing si indica un insieme di tecnologie che permettono,
  tipicamente sotto forma di un servizio offerto da un provider al
  cliente, di memorizzare/archiviare e/o elaborare dati (tramite CPU o
  software) grazie all'utilizzo di risorse hardware/software
  distribuite e virtualizzate in Rete.  La correttezza nell'uso del
  termine è contestata da molti esperti: se queste tecnologie sono
  viste da alcuni analisti come una maggiore evoluzione tecnologica
  offerta dalla rete Internet, da altri, come Richard Stallman, sono
  invece considerate una trappola di marketing.'', tratto da
  Wikipedia: \url{http://it.wikipedia.org/wiki/Cloud_computing}.}. Ai
protocolli standard si sostituiscono API proprietarie e suscettibili
di alterazioni continue e unilaterali. Un'impresa offre servizi
tramite API, decidendo i termini con cui gli utenti, o altre imprese,
possono accedere e utilizzare questi servizi.

Dal sito di Google:
\begin{quote}
  ``\emph{Google has been pushing the technological bounds of cloud
  computing for more than ten years. Today, feedback and usage
  statistics from hundreds of millions of users in the real world help
  us bring stress-tested innovation to business customers at an
  unprecedented pace. From our consumer user base, we quickly learn
  which new features would be useful in the business context, refine
  those features, and make them available to Google Apps customers
  with minimal delay.}''\footnote{``Google ha spinto i limiti
    tecnologici del cloud computing da più di dieci anni. Oggi, i
    feedback e le statistiche di uso su centinaia di milioni di utenti
    reali nel mondo ci aiutano a portare innovazioni ben testate ai
    nostri clienti business ad un ritmo senza precedenti. Dalla nostra
    base di utenti, apprendiamo velocemente quali caratteristiche
    possono essere utili per il mercato, migliorandole e rendendole
    disponibili agli utenti di Google Apps con un ritardo minimo'',
    Tratto da
    \url{http://www.google.com/apps/intl/en/business/cloud.html}.}
\end{quote}

Come dichiarato inoltre, ``il Cloud computing è nel DNA di Google''
che vanta già 350 milioni di utenti attivi nella sua
nuvola.\footnote{Dichiarazione rilasciata sul ``earning call'' del 19
  gennaio 2012 vedi:
  \url{http://thenextweb.com/google/2012/01/19/gmail-closes-in-on-hotmail-with-350-mm-active-users/}.}.

Secondo la società di consulenza Gartner, nel 2016, tutte le compagnie
contemplate dal \emph{Forbes Global 2000} faranno uso di soluzioni
\emph{cloud} \citep{EY2011}.

In sostanza un tempo lo sviluppo seguiva un modello \emph{bottom up}, per
cui le maestranze informatiche sviluppavano sistemi ad hoc per le
esigenze locali, per poi in seguito aprire una discussione in rete,
con i loro corrispettivi, per definire dei protocolli standard e
mettere in comunicazione il tutto.

Oggi si passa ad un modello di sviluppo \emph{top down}, per cui nuovi
servizi vengono lanciati basandosi su indagini di mercato e test su
campioni di utenti. Poche imprese fanno da padrone nel tracciare lo
sviluppo della rete che assume orizzonti sempre più determinati dal
mercato. Inoltre il mercato di riferimento, e il campione scelto, sono
per lo più legati al contesto economico, sociale e culturale
nordamericano e europeo. 

Pur essendo un'analisi di interesse filosofico/antropologico diventa
territorio di frontiera con l'informatica studiare la cultura digitale
e gli effetti che le tecnologie digitali possono avere sulle culture
di paesi e popoli diversi. Una analisi strutturata e analitica ci è
stata lasciata da Vilém Flusser, recentemente riscoperto e considerato
tra i primi filosofi della comunicazione dei nostri tempi, di cui
riporto due pensieri molto attuali, anche se pubblicati già nel 1983
nel libro \emph{F{\"u}r eine Philosophie der Fotografie}:

\begin{quote}
  ``\emph{Both those taking snaps and documentary photographers, however,
  have not understood 'information.' What they produce are camera
  memories, not information, and the better they do it, the more they
  prove the victory of the camera over the human
  being.}''\footnote{``Sia chi si limita a scattare foto, sia i fotografi
documentaristi, non hanno compreso l' 'informazione'. Quello che
producono sono memorie fotografiche, non informazione, e meglio lo
fanno, più confermano la vittoria della macchina fotografica
sull'essere umano''}, \citet{flusser1983philosophie}.
\end{quote}

e 

\begin{quote}
  ``\emph{Our thoughts, feelings, desires and actions are being
    robotized; 'life' is coming to mean feeding apparatuses and being
    fed by them. In short: everything is becoming absurd. So where is
    there room for human freedom?}''\footnote{``I nostri pensieri,
    sentimenti, desideri e azioni vengono robotizzati; la ``vita''
    comincia a significare nutrire apparati e essere nutriti da
    essi. In breve: tutto sta diventando assurdo. Dov'è dunque lo
    spazio per la libertà umana?''}, \citet{flusser1983philosophie}.
\end{quote}

Un approccio diametralmente opposto alla centralizzazione della rete
sono le tecnologie Peer To Peer (P2P)\footnote{Il principio base del
  P2P prevede una infrastruttura di comunicazione in cui i nodi
  comunicano direttamente tra di loro senza nodi centrali
  preconfigurati. Esistono anche protocolli P2P che prevedono dei
  super-nodi, che in parte riprendono il modello client/server,
  normalmente usati per ottimizzare le prestazioni e risolvere
  problemi di attraversamento di reti mascherate. In generale i
  sistemi P2P sono auto-configuranti o necessitano di una conoscenza
  minima dell'architettura della rete sottostante, dato che si basano
  su protocolli per l'instradamento automatico dei pacchetti. Queste
  tecnologie funzionano bene quando un numero sufficiente di nodi è
  attivo e partecipa al funzionamento della rete. Un esempio di
  protocollo P2P molto diffuso è bittorrent, con cui è possibile
  trasferire dati ad alte velocità quando i contenuti richiesti sono
  presenti su nodi con banda a disposizione.}.

Nonostante le tecnologie P2P siano un'alternativa disponibile e già
implementata a vari livelli, presentano alcune caratteristiche per cui
non possono essere ampiamente applicabili al contesto della Rete
Mocambos. Le connessioni tra le comunità della RM sono tutte via
satellite, con banda molto limitata. I protocolli P2P, gravando sulla
banda di molti nodi, per operazioni effettuate da un singolo nodo,
penalizzerebbero proprio la risorsa più scarsa. 

Le necessità di una rete, come la Rete Mocambos, sono di vario tipo e
gestite da entità differenti sebbene federate. Nel prossimo capitolo
vedremo un tipo di rete federata per la RM.

