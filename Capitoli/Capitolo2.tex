% Capitolo 2

\chapter{Reti federate eventualmente connesse}
\label{Capitolo2}
\lhead{Capitolo 2. \emph{Reti federate eventualmente connesse}}

\section{Le Reti federate}
Con rete federata si può intendere un universo molto ampio di
situazioni. In questo lavoro si ricerca sopratutto una rete federata
con i seguenti aspetti:
\begin{itemize}
  \item amministrazione decentrata 
  \item accesso alla gestione logica (e fisica) della rete
  \item conoscenze tecniche locali sulle tecnologie usate dalla rete
  \item servizi interni alla rete federata
  \item interazione tra reti locali intelligenti
\end{itemize} 

Una rete federata quindi intesa come congiunto di soluzioni
tecnicamente viabili ed adattabili ad usi, costumi e contesti
differenti. Permette una gestione flessibile della rete con sotto-reti
eterogenee sfruttando diverse tecnologie, ad esempio, P2P ove
necessario. Si adatta ad un contesto dove esiste già una struttura
organizzativa che si può riflettere nella struttura di rete e che ne
può accompagnare la gestione.

\subsection{Reti federate eventualmente connesse}
L'introduzione sul contesto suggerisce l'ambito di ricerca ma
necessita una restrizione maggiore. Per ``rete
federata eventualmente connessa'', consideriamo, oltre alle premesse
fatte, principalmente gli aspetti legati ad una rete federata basata
su connessioni non sempre disponibili, quali le connessioni
satellitari.

\subsection{La Rete Mocambos (RM)}
\label{sec:ReteMocambos}
``La "Rete Mocambos" è una rete solidaria di comunità che ha come
principale obbiettivo condividere idee e offrire appoggio reciproco''.
``La tecnologia è uno dei campi di attuazione della Rete essendo al
tempo stesso idea e mezzo per condividere idee''. \footnote{Tradotto da
  \url{http://www.mocambos.org/sobre}.}

La RM attualmente coinvolge direttamente più di 200 comunità
quilombola, collettivi, comunità indigene e terreiros. Esistono due
accordi tra la Rete Mocambos \footnote{L'accordo sono stati firmati
  come azione della Rete Mocambos dalla ``Casa de Cultura Tainã''. La
  ``Casa de Cultura Tainã'' è un'entità sociale e culturale senza fini
  di lucro fondata nel 1989 da abitanti della periferia di
  Campinas. Obbiettivo dell'entità è rendere possibile l'accesso
  all'informazione fortificando la pratica di cittadinanza e la
  formazione dell'identità culturale, al fine di contribuire nella
  crescita di individui coscienti e attuanti nella
  comunità. L'indirizzo del sito internet è
  \url{http://www.taina.org.br}.} i il Ministero delle Comunicazioni
all'interno del programma GESAC e del programma Telecentros.BR. Il
GESAC garantisce connettività satellitare a tutte le comunità con una
banda larga limitata dovuta al tipo di tecnologia, non facilmente
sostituibile nel breve e medio termine. Tramite il programma
Telecentros.BR sono in corso di installazione 200 "telecentros" (sale
attrezzate con 10 computer ad accesso pubblico) e 200 borse per tutor
locali, per la gestione degli spazi. Proprio in un contesto cosi
specifico nasce la necessità di adattare la tecnologia alle esigenze
locali imposte, anche per le limitazioni tecniche. Questa limitazione
spinge a riconsiderare la rete non solo come il mezzo di connessione
verso i grandi \emph{datacenter}. La rete può, e in questo caso deve,
essere strutturata nel territorio con logiche di sviluppo e gestione
localmente determinate dalla comunità. In questo senso è
indispensabile la formazione e l'accesso alle tecnologie. La
\emph{Casa de Cultura Tainã}, nucleo fondatore della RM, è tra le
prime realtà popolari che hanno percepito la necessità del software
libero, come espressione della libertà di poter creare i propri
strumenti tecnologici digitali. L'esclusione digitale non riguarda
solo l'accesso al mondo digitale ma l'impossibilità di contribuire
alla sua creazione e crescita.

Per la RM è importante, sotto molti aspetti, poter costruire e gestire
i mezzi di comunicazione e adattarli al loro contesto. In Brasile,
sono molte le comunità indigene, quilombola e ``tradizionali''
\footnote{Per comunità tradizionali si intende comunità con culture e
  società proprie, per lo più di origine afro-indigena, situate spesso
  in ambito rurale e fortemente legate alla natura. In Brasile queste
  comunità hanno riconosciuto costituzionalmente il diritto di vivere
  secondo la loro cultura.} che conoscono bene le potenzialità delle
tecnologie digitali e stanno, a poco a poco, cominciando a
dominarle. Questo non sarebbe stato possibile senza l'esistenza e
l'ampia disponibilità di Software Libero. Tecnologie digitali sotto
forma di prodotti commerciali sarebbero un ennesimo passo verso la
dipendenza economica e culturale. Questa consapevolezza è alla base di
scelte ben ponderate. Ricordo che qualche anno fa ad una riunione, a
Brasilia, del programma del governo federale ``Luz para Todos''
(programma per fornire elettricità anche alla popolazione in zone
rurali e amene) un capo indigena fece presente che pur volendo
l'energia elettrica questa dovesse essere limitata solo agli spazi
comunitari e non alle case. Non è difficile intuire il perché di tale
condizione. Portare l'elettricità nelle case non è compatibile con la
loro cultura e nemmeno con la loro economia. Ad un contatore elettrico
corrisponde una bolletta che richiede denaro, che porterebbe alla
migrazione in cerca di lavoro retribuito al di fuori della comunità.

\section{DA RIVEDERE}
Una rete federata deve essere gestita. Servono risorse e conoscenze
locali. Questo comporta un investimento strategico che riguarda
aspetti ecomomici, ma sopratutto umani. E' necessaria una formazione
continua e personale preparato per la gestione. Il mercato ha escluso
questo modello anche per i suoi costi. Gli attori principali della
new-economy globale sono riusciti a rendere questo passaggio quasi
naturale, offrendo sistemi altamente funzionali e a condizioni
particolarmente appetibili. Inanzitutto i servizi sono gratuiti e
questo è la base per renderne l'uso in massa possibile. Da un punto di
vista tecnico-informatico, la diffusione di servizi internet modello
ToS vincenti, ha provocato una migrazione in massa delle maestrie del
settore, sopratutto le nuove generazioni, verso lo sviluppo basato su
API et similia.

Passando dai protocolli di comunicazione standard alle API/ToS, dalla
gestione della rete all'accesso al cloud, si p le conoscenze e
sopratutto la ricerca

(RITORNARE A DIRE PERCHE LE RETI FEDERATE SONO INTERESSANTI. SERVE UNA
SCELTA POLITICA PER RIPRENDERE IL POSSESSO DELLA RETE COME SPAZIO
TECNOLOGICO NON SOLO COME CONNESSIONE A POCHI CLOUD E DATACENTER)
. Inoltre la programmazione di rete via API ha reso gli orizzonti in
buona parte predeterminati dalle imprese madri del servizio.

I nuovi mezzi di comunicazione offrono evidenti vantaggi, a cui non
possiamo più rinunciare, tanto da spingere le Nazioni Unite a
riconoscere l'accesso a internet come un diritto fondamentale \ref{}.

Consideriamo in particolare una rete di servizi federati tra molte
reti locali (attualmente circa 200 per la Rede Mocambos). La
restrizione, oltre a essere dettata da un vincolo reale (la Rede
Mocambos è basata su connessioni satellitari), può rappresentare un
punto di partenza per ampliare la ricerca verso reti locali
intelligenti e più sostenibili. Anche la diffusione di reti mesh ha
risvegliato l'interesse per le reti di servizi federati su risorse
decentralizzate e gestite localmente.


\section{Tecnologie}

Prima di addentrarci nei requisiti specifici, e nelle scelte adottate,
può essere utile una breve rassegna degli strumenti tecnologici presi
in considerazione per strutturare il prototipo per la Rede Mocambos di
rete federata eventualmente connessa, in particolare per i suoi
servizi di base, quali identificazione, autenticazione e
messaggistica.

\subsection{LDAP}
Lightweight Directory Access Protocol (LDAP) è un insieme di
protocolli aperti per accedere a informazioni conservate centralmente
attraverso una rete. LDAP organizza le informazioni attraverso una
gerarchia ad albero chiamata Directory Information Tree (DIT). LDAP è
un sistema client/server. Il server può usare una varietà di database
per conservare un DIT, normalmente ottimizzati per le operazioni di
lettura. Quando un'applicazione client si collega ad un server LDAP,
può sia interrogare la directory che cercare di modificarla. Nel caso
in cui si verifica una interrogazione, il server può rispondere in
modo locale, oppure può inoltrare la richiesta ad un server LDAP che
sia in possesso di una risposta. Se l'applicazione di un client stà
cercando di modificare le informazioni all'interno di una directory
LDAP, il server verifica se l'utente possiede il permesso di
effettuare il cambiamento, e successivamente aggiunge o aggiorna le
informazioni. LDAP supporta la delega di parte del DIT a server
specifici, la replica in sola lettura e la replica in
lettura/scrittura (multi-master).  LDAP è un protocollo solido, molto
diffuso e supportato, e da tempo è lo standard de facto per gestire
basi di dati di utenti. OpenLDAP è una implementazione aperta e libera
del protocollo LDAP, che include client, server e una serie di
strumenti per facilitarne l'amministrazione.

\subsection{XMPP}
Extensible Messaging and Presence Protocol (XMPP) è un insieme di
protocolli aperti per la messaggistica e la presenza in rete basato su
XML. XMPP è un sistema client/server. Le specifiche per la
comunicazione tra server fanno si che gli utenti di un server possono
interagire in modo trasparente con gli utenti di altri server
federati. La XMPP Standards Foundation (XSF), coordina lo sviluppo
delle estensioni dello standard tramite le XMPP Extension Protocols
(XEPs), che ad oggi sono ben 311. XMPP e le XEPs costituiscono una
ambiente flessibile e completo per lo sviluppo di servizi
federati. Questi protocolli sono gia utilizzabili grazie a moltissime
implementazioni di server, client e librerie liberi. Anche la storia
di XMPP, un tempo noto come Jabber, è interessante. Jabber venne
infatti inizialmente sviluppato da Jeremie Miller nella sua fattoria
nell'Iowa. E' un esempio concreto di come la ricerca e lo sviluppo di
tecnologie della comunicazione, fuori da ambiti accademici e
imprenditoriali, può essere proficua. Oggi infatti XMPP è la
tecnologia più usata per la messaggistica anche dai grandi attori
della new economy. Per le necessità della propria rete è possibile
quindi estendere le funzionalità di un server XMPP e al tempo stesso
usufruire dei servizi base standard (messagistica e presenza),
implementanti dai server già esistenti in rete.

\subsection{OpenID}
OpenID è un sistema di identificazione decentralizzato nel quale la
propria identità è un URL che può essere verificata da qualunque
server supporti il protocollo. E' un protocollo aperto e sono
disponibili varie implementazioni libere. Inoltre il protocollo è
stato adottato dai principali fornitori di servizi web. Con OpenID è
possibile usare la stessa identità su piu servizi ed è un ottima base
per un sistema Single Sign On (SSO). Il protocollo sfrutta HTTP e e
Cookies per mantenere una sessione attiva. Al primo tentantivo di
autenticazione presso un servizio compatibile con OpenID, si viene
reindirizzati verso il proprio provider OpenID per effetturare
l'accesso e confermare l'autorizzazione a procedere al servizio
iniziale. Per tutta la durata della sessione, è possibile accedere ai
servizi OpenID senza reinserire le credenziali.

\subsection{OAuth}
OAuth è un protocollo aperto per l'autorizzazione di servizi tramite
API. Ad esempio, permette ad un utente di dare l'accesso alle sue
informazioni presenti su un sito detto service provider, ad un altro
sito, chiamato consumer, senza però condividere la sua identità. E' un
metodo per pubblicare e interagire con dati protetti. Esistono molti
altri protocolli e API simili come Google AuthSub, AOL OpenAuth, Yahoo
BBAuth, Upcoming API, Flickr API, Amazon Web Services e ognuno
fornisce dei metodi proprietari per lo scambio di credenziali e per
l'accesso tramite token. (NOTA) OAuth è una standardizzazione aperta
delle pratiche piu diffuse. Inoltre è stato pensato per il supporto a
vari tipi di applicazioni, non soltanto per i servizi web.


\subsection{Shibboleth}
Shibboleth è un architettura e un implementazione aperta per
l'autenticazione e autorizzazione di identità federate basata su
Security Assertion Markup Language (SAML). Le identità federate
permettono che le informazioni di un utente sotto un certo dominio
possano essere condividise con un altro dominio federato. Questo
permette il SSO attraverso piu domini senza lo scambio di nomi utente
e password. Gli IdP mantengono le informazioni sull'utente mentre i SP
fanno uso di queste informazioni per l'accesso sicuro ai contenuti.

\subsection{Kerberos}
Kerberos è un protocollo aperto per l'autenticazione forte in reti di
computer. E' un protocollo client/server e consente l'autenticazione
reciproca, ossia entrambi verificano le loro identità. Kerberos è
basato sul protocollo di Needham-Schroeder a chiavi simmetriche e
prevede un entità terza di fiducia chiamata Key Distribution Center
(KDC).


Noosfero

Netsukuku


Parlare di netsukuku oslr e reti mesh 

TCP/IP - flag day 1 gen 1983 (prima NCP) - RFC 675
http://tools.ietf.org/html/rfc675

Dal Request For Comment si passa al Terms Agreements.
