% Capitolo 4

\chapter{Un prototipo di servizio federato}
\label{Capitolo4}
\lhead{Capitolo 4. \emph{Un prototipo di servizio federato}}

L'uso delle tecnologie digitali offre nuove possibilità per
l'insegnamento. Molte comunità, afferenti alla RM, hanno iniziato la
produzione di materiale pedagogico audio-visuale, che con l'aiuto
della rete, potrebbe contribuire ad arricchire i programmi educativi
pubblici spesso carenti e poco integrati con la cultura quilombola. Il
\emph{Núcleo de Pesquisa e Desenvolvimento Digital (NPDD)}\footnote{Il
  \emph{Núcleo de Pesquisa e Desenvolvimento Digital (NPDD)} della RM
  ricerca e sviluppa tecnologie digitali per la comunicazione, la
  produzione di energie rinnovabili e sostenibili, e il miglioramento
  delle condizioni di vita in simbiosi con l'ambiente. Maggiori
  informazioni su \url{http://wiki.mocambos.net/wiki/NPDD}.}, nucleo
di ricerca e sviluppo digitale della RM, con il progetto \emph{Tambor
  e Comunicação}\footnote{Il progetto \emph{Tambor e Comunicação} é un
  tentativo di fortificare la rete di comunicazione digitale per le
  necessità delle comunità. Vedi
  \url{http://wiki.mocambos.net/wiki/Projeto_Tambor_e_Comunicacao}.},
ha proposto la ricerca e sviluppo di una soluzione per la
pubblicazione e diffusione in rete di immagini, audio e video di
interesse comune e spesso prodotti nelle stesse comunità.


\section{Sistema di pubblicazione e diffusione di contenuti
  multimediali}
Il sistema sviluppato prevede l'installazione di un portale sul server
locale delle comunità attraverso il quale è possibile visualizzare e
pubblicare contenuti multimediali sfruttando l'alta velocità della
rete locale. Il sistema si prende cura di memorizzare i contenuti in
un archivio multimediale locale e diffondere, quelli etichettati come
di interesse comune, verso i server delle altre comunità (vedi figura
\ref{fig:SchemaServer_ReteMocambos}). Il prototipo sviluppato cerca di
risolvere le limitazioni di banda della rete rispettando la specifica
di requisiti. 

\begin{figure}[htbp]
  \centering
  \includegraphics[width=\textwidth]{./Figure/SchemaServer_ReteMocambos-crop.pdf}
  \rule{35em}{0.5pt}
  \caption[Schema dell'infrastruttura della RM]{Schema dell'infrastruttura della RM.}
  \label{fig:SchemaServer_ReteMocambos}
\end{figure}


\section{Archivio multimediale}
L'archivio multimediale locale della comunità è un \emph{repository
  git-annex} che viene gestito tramite il portale comunitario, sia per
la pubblicazione che per la visualizzazione dei contenuti. L'accesso
diretto ai dati è comunque garantito essendo i dati salvati in chiaro
sul disco. Inoltre è possibile interagire con i dati attraverso
l'ampio universo di applicazioni e librerie di \emph{git}. In
particolare il prototipo utilizza la struttura dei metadati di
\emph{git} per mantenere traccia del responsabile della pubblicazione
di un contenuto. L'operazione di \emph{commit} coincide in effetti con
la pubblicazione di un contenuto, e il \emph{committer} con l'utente
che lo pubblica. I metadati, invece, relativi al contenuto
multimediale in se, quali autore, tipo di licenza, data di creazione,
possono essere memorizzati all'interno del file seguendo gli standard
esistenti come il \emph{Dublin Core}\footnote{``Il Dublin Core è un
  sistema di metadati costituito da un nucleo di elementi essenziali
  ai fini della descrizione di qualsiasi materiale digitale
  accessibile via rete informatica'', tratto da Wikipedia:
  \url{http://it.wikipedia.org/wiki/Dublin_Core}.}. 

\subsection{git-annex}
\emph{git-annex}\footnote{\emph{git-annex} è un programma che estende
  le funzionalità di \emph{git} ed è disponibile su
  \url{http://git-annex.branchable.com}.} permette la gestione di file
con \emph{git}, senza la necessità di aggiungere il file dentro
\emph{git}. Anche se può sembrare paradossale, è utile quando si ha a
che fare con file molto grandi che \emph{git} attualmente non può
gestire facilmente per limitazioni dovute a memoria, tempo o spazio
nel disco.

Anche senza mantenere traccia del contenuto del file, avere la 
possibilità di gestire i file con \emph{git}, di spostarli e cancellarli su un
albero di cartelle versionato, con uso di \emph{branches} e di cloni
distribuiti, sono tutti buoni motivi per usare \emph{git}. E i file allegati
(da cui il nome \emph{git-annex}) possono coesistere nello stesso repository
\emph{git} con i file regolarmente versionati.

\emph{git-annex} trasforma i file allegati in \emph{link}
simbolici, che vengono normalmente versionati da \emph{git}. 

Il contenuto dei file viene mantenuto da \emph{git-annex} in un
archivio chiave/valore distribuito che corrisponde ai cloni di un
dato \emph{repository git}. Praticamente \emph{git-annex} memorizza il
contenuto del file in una sotto-cartella di \verb|.git/annex/|.

La prima volta che un file viene aggiunto a \emph{git-annex}, viene
calcolata una chiave, normalmente facendo un \emph{hash} del suo
contenuto. \emph{git-annex} tuttavia supporta diversi \emph{backend}
che possono produrre vari tipi di chiavi. Il file che viene aggiunto a
\emph{git} non è altro che un \emph{link} simbolico alla chiave
memorizzata in \verb|.git/annex/|. Se il contenuto del file viene
modificato, viene prodotta un'altra chiave, e il \emph{link} viene
cambiato.

Il contenuto del file può essere trasferito da un \emph{repository}
all'altro da \emph{git-annex}, che inoltre mantiene traccia di chi
contiene cosa, permettendo ad esempio di creare una mappa delle copie
disponibili e impostare il numero di copie minime. Queste informazioni
vengono mantenute su un \emph{branch} separato, chiamato
``\emph{git-annex}'', e le operazioni di sincronizzazione non sono
altro che \emph{push} e \emph{pull} tra i vari \emph{repository}.

\emph{git-annex} supporta:
\begin{itemize}
\item localizzazione delle copie (\emph{location tracking})
\item scaricamento selettivo dei contenuti
\item gestione della fiducia dei \emph{repository}
\item gestione del numero di repliche minimo 
\item vari \emph{backend} per le chiavi (SHA, WORM)
\item vari \emph{backend} per i contenuti/valori (bup, rsync, web, S3)
\end{itemize}

\section{Portale Comunitario}
\framebox[\textwidth]{\footnotesize Il codice è disponibile su
\url{https://github.com/RedeMocambos/Mocambos_Portal_Local}}

Il portale locale deve dare accesso ai principali servizi locali della
comunità. Per lo sviluppo è stato scelto l'uso di un framework basato
su python, \emph{Django}, che consente un'integrazione flessibile e
avanzata con altri sistemi grazie alle numerose librerie disponibili
quale, ad esempio, quella per l'autenticazione LDAP.

Per gestire l'archivio multimediale \emph{git-annex} sono stati
sviluppati due moduli per \emph{Django}, che definiscono il modello
dei dati e si prendono cura di aggiungere i contenuti sul
\emph{repository} eseguendo le operazioni di \emph{commit},
\emph{push} e \emph{pull}. 

\begin{figure}[htbp]
  \centering
  \includegraphics[width=\textwidth]{./Figure/UML_Schema_Django-crop.pdf}
  \rule{35em}{0.5pt}
  \caption[Schema UML dei moduli django]{Schema UML dei moduli django.}
  \label{fig:SchemaUMLDjango}
\end{figure}

\subsection{Django mmedia}
\framebox[\textwidth]{\footnotesize Il codice è disponibile su
\url{https://github.com/RedeMocambos/mmedia}}

Il modulo \emph{mmedia} implementa il modello dei dati
multimediali con supporto al salvataggio su \emph{repository
  git-annex}.

I modelli, in Django, devono essere creati nel file \verb|model.py|,
dove, innanzitutto, definiamo la struttura base della classe
\emph{MMedia}, con gli attributi comuni a tutti gli oggetti
multimediali. Le classi \emph{Audio}, \emph{Image} e \emph{Video},
ereditano dalla classe astratta ``MMedia'', includendo altri attributi
specifici per il tipo di oggetto.

Gli oggetti vengono salvati su \emph{database}, e serializzati su
disco tramite l'\emph{overriding} della funzione \emph{save()}:

\begin{code}
    def save(self, *args, **kwargs):
        logger.debug(type(self))
        serializeTo = os.path.join(settings.MEDIA_ROOT,\
                                   settings.GITANNEX_DIR,\
                                   settings.PORTAL_NAME,\
                                   settings.SERIALIZED_DIR,\
                                   os.path.basename(self.fileref.path)+ '.xml')
        logger.info('>>>> Serialize to: ' + serializeTo)
        out = open(serializeTo, "w")
        XMLSerializer = serializers.get_serializer("xml")
        xml_serializer = XMLSerializer()
        xml_serializer.serialize((self, ), stream=out)
        super(MMedia, self).save(*args, **kwargs)
\end{code}

\subsection{Django gitannex}
\framebox[\textwidth]{\footnotesize Il codice è disponibile su
\url{https://github.com/RedeMocambos/gitannex}}

Il modulo \emph{gitannex} implementa parte del modello dei dati di un
\emph{repository git-annex} in Django, aggiungendo gli attributi e le
funzionalità necessarie alla programmazione della sincronizzazione.

\begin{figure}[htbp]
  \centering
  \includegraphics[width=\textwidth]{./Figure/SequenceDiagram_NuovoOggetto-crop.pdf}
  \rule{35em}{0.5pt}
  \caption[Diagramma di sequenza della creazione di un nuovo oggetto
  multimediale]{Diagramma di sequenza della creazione di un nuovo
    oggetto multimediale.}
  \label{fig:SequenceDiagramAdd}
\end{figure}

Seguendo la specifica, viene definito, tramite l'attributo
\emph{syncStartTime}, un orario per l'inizio della sincronizzazione,
che viene lanciata dalla funzione \emph{runScheduledJobs()}.

Django fornisce un sistema di segnali, lanciati in concomitanza di
operazioni quali la \emph{save()} di un oggetto, che possono essere
intercettati altrove dal sistema. Il modulo \emph{gitannex} intercetta
il segnale standard di Django, \emph{post-save} (vedi figura
\ref{fig:SequenceDiagramAdd}), su oggetti che ereditano dalla classe
\emph{MMedia}\footnote{Django non supporta il ``filtraggio'' dei
  segnali lanciati dalle sottoclassi di una data classe, in questo
  caso \emph{MMedia}. A tal fine, è stato usato un trucco reperito in
  rete (vedi il codice nel file \texttt{gitannex/signals.py}).},
tramite la funzione \emph{gitMMediaPostSave()}:


\begin{code}
@receiver_subclasses(post_save, MMedia, ``mmedia_post_save'')
def gitMMediaPostSave(instance, **kwargs):
    logger.debug(instance.mediatype)
    logger.debug(type(instance))
    logger.debug(instance.path_relative())

    path = instance.path_relative().split(os.sep)
    if gitannex_dir in path:
        repositoryName = path[path.index(gitannex_dir) + 1]
        gitAnnexRep = GitAnnexRepository.objects.get(\
                      repositoryName__iexact=repositoryName)
        gitAnnexAdd(os.path.basename(instance.fileref.name),\
                    os.path.dirname(instance.fileref.path))
        gitCommit(instance.title, instance.author.username,\
                  instance.author.email, os.path.dirname(instance.fileref.path))
\end{code}







