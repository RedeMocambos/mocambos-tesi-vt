% Capitolo 4

\chapter{Un prototipo di servizio federato}
\label{Capitolo3}
\lhead{Capitolo 3. \emph{Un prototipo di servizio federato}}

\section{Sistema di pubblicazione e diffusione di contenuti
  multimediali}
L'uso delle tecnologie digitali offre nuove possibilità per
l'educazione. Molte comunità afferenti alla RM hanno iniziato la
produzione di materiale pedagogico audio-visuale che con l'aiuto della
rete potrebbe contribuire ad arricchire i programmi educativi pubblici
spesso carenti e poco integrati con la cultura quilombola. Da tempo
\footnote{Il progetto \emph{Tambor e Comunicação} é un tentativo di
  fortificare la rete di comunicazione digitale per le necessità delle
  comunità. Vedi
  \url{http://wiki.mocambos.net/wiki/Projeto_Tambor_e_Comunicacao}.}
quindi il Nucleo di Ricerca e Sviluppo Digitale (NPDD) della RM
\footnote{Il \emph{Núcleo de Pesquisa e Desenvolvimento Digital} della
  RM ricerca e sviluppa tecnologie digitali per la comunicazione, la
  produzione di energie rinnovabili e sostenibili, e il miglioramento
  delle condizioni di vita in simbiosi con l'ambiente. Maggiori
  informazioni su \url{http://wiki.mocambos.net/wiki/NPDD}.} cerca una
soluzione per la pubblicazione e diffusione in rete di immagini, audio
e video di interesse comune e spesso prodotti nelle stesse
comunità. Il sistema prevede l'installazione di un portale sul server
locale delle comunità su cui è possibile pubblicare contenuti
multimediali sfruttando l'alta velocità della rete locale. Il sistema
si prende cura di diffondere i contenuti, etichettati come di
interesse comune, verso i server delle altre comunità. Il prototipo
sviluppato cerca di risolvere le limitazioni di banda della rete
rispettando la specifica di requisiti.

\section{Portale Comunitario}
Per lo sviluppo di un portale locale è stato scelto l'uso di Django.


