% Capitolo 3

\chapter{Architettura e prototipo per la Rede Mocambos
} % Write in your own chapter title
\label{Capitolo3}
\lhead{Capitolo 3. \emph{Architettura e prototipo per la Rede Mocambos
}} % Write in your own chapter title to set the page header

\subsection{Specifica dei requisiti}

La Rede Mocambos attualmente è composta da piu di 200 comunità
localizzate in tutto il territorio nazionale brasiliano e alcune
comunità in Africa e in Europa. La connettività in Brasile è
satellitare grazie al programma GESAC che mette a disposizione di ogni
comunità un Very Small Aperture Terminal (VSAT) con banda di 512
kbit/s in download e 128 kbit/s in upload. La topologia della rete è a
stella per cui tutti i nodi comunicano tramite satellite concentrando
il traffico su un hub terrestre, dove la rete satellitare è
interconnessa a Internet. Ogni comunità ha a disposizione una sala con
10 computer ad accesso pubblico (chiamati comunemente
``telecentros''), installati recentemente tramite il Telecentros.BR
\footnote{``Il programma nazionale di appoggio all'inclusione digitale
nelle comunità, Telecentros.BR, è un'azione del Governo Federale di
appoggio per la creazione di nuovi spazi pubblici e comunitari di
inclusione digitale e per il potenziamento degli spazi già in funzione
su tutto il territorio. Vengono messe a disposizione le
apparecchiature informatiche e il mobiliario necessario al
funzionamento dei \textit{telecentros}, servizi di connessione a
internet tramite banda larga oltre a formazione e borse per l'appoggio
economico per tutor che lavorano come agenti di inclusione
digitale. Questi tutor borsisti partecipano ad un corso di formazione
e lavorano, nei telecentros, per le comunità.'', tradotto da
http://www.inclusaodigital.gov.br/telecentros}, altro programma
federale brasiliano.

