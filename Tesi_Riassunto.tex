%% ----------------------------------------------------------------
%% Thesis.tex -- MAIN FILE (the one that you compile with LaTeX)
%% ---------------------------------------------------------------- 

% Set up the document
\documentclass[a4paper, 12pt, oneside]{Tesi_Riassunto}  % Use the "Thesis" style, based on the ECS Thesis style by Steve Gunn
\graphicspath{{Figure/}}  % Location of the graphics files (set up for graphics to be in PDF format)



% Include any extra LaTeX packages required
\usepackage[utf8]{inputenc}
\usepackage[italian]{babel}
\usepackage[T1]{fontenc}
\usepackage{ae,aecompl}
\usepackage[square, numbers, comma, sort&compress]{natbib}  % Use the "Natbib" style for the references in the Bibliography
\usepackage{verbatim}  % Needed for the "comment" environment to make LaTeX comments
\usepackage{vector}  % Allows "\bvec{}" and "\buvec{}" for "blackboard" style bold vectors in maths
%\usepackage{listings}
\usepackage{listingsutf8}
\lstset{frame=tb,
  language=Python,
  aboveskip=3mm,
  belowskip=3mm,
  showstringspaces=false,
  columns=flexible,
  basicstyle={\small\ttfamily},
  numbers=none,
  numberstyle=\tiny\color{gray},
  keywordstyle=\color{dkblue},
  commentstyle=\color{dkgreen},
  stringstyle=\color{mauve},
  breaklines=true,
  breakatwhitespace=true
  tabsize=2
  morekeywords={models, lambda, forms, =}
  inputencoding=utf8,
  extendedchars=\true
}

\lstnewenvironment{code}[1][]%
  {\minipage{\linewidth} 
   \lstset{basicstyle=\ttfamily\footnotesize,frame=tb,#1}}
  {\endminipage}


%%%% Color definitions
\usepackage{color}
\definecolor{dkgreen}{rgb}{0,0.6,0}
\definecolor{dkblue}{rgb}{0,0,0.5}
\definecolor{gray}{rgb}{0.5,0.5,0.5}
\definecolor{mauve}{rgb}{0.6,0,0.82}
\definecolor{coolblack}{rgb}{0.0, 0.18, 0.39}
\hypersetup{urlcolor=coolblack, colorlinks=true}  % Colours hyperlinks in blue, but this can be distracting if there are many links.
%\hypersetup{urlcolor=black, colorlinks=false}  

%% To show a border around all figures
%\usepackage{float}
%\floatstyle{boxed} 
%\restylefloat{figure}

%% ----------------------------------------------------------------
\begin{document}
\frontmatter	  % Begin Roman style (i, ii, iii, iv...) page numbering

% Set up the Title Page
\title  {Reti federate eventualmente connesse}

\authors  {\texorpdfstring
            {\href{mailto:vince@mocambos.net}{Vincenzo Tozzi}}
            {Vincenzo Tozzi}
            }
\addresses  {\groupname\\\deptname\\\univname}  % Do not change this here, instead these must be set in the "Thesis.cls" file, please look through it instead
\date       {\today}
\subject    {Tesi di Laurea in Informatica}
\keywords   {Federated Network, Tecnological Autonomy, Free Software, Rede Mocambos, Casa de Cultura Taina, Quilombo}


% The Abstract Page
%\addtotoc{Abstract}  % Add the "Abstract" page entry to the Contents
\abstract{

  Questa tesi, che nasce da un'esperienza diretta, ricerca tecnologie
  per una rete federata di comunità afro-discendenti incentrata
  sull'autonomia e sulla libertà. Il lavoro propone un'architettura e
  un prototipo per una ``rete federata eventualmente connessa'' i cui
  vincoli tecnologici principali sono la banda e la disponibilità
  della connessione. Il prototipo è un sistema per la condivisione di
  contenuti multimediali tra portali web locali, basato su
  \emph{Django} e \emph{git-annex}, autenticati su server \emph{LDAP}.

}

\clearpage  % Abstract ended, start a new page

\end{document}  % The End
%% ----------------------------------------------------------------
